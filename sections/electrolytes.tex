\documentclass[../main.tex]{subfiles}

\begin{document}

\section{Electrolytes}
\label{sec:electrolytes}
\subsection{Introduction (Arihant)}% and historical context}
\label{sec:introduction_electrolytes}
An important challenge for development of Li-ion rechargeable batteries for electric vehicles is: development of a non-flammable safe electrolyte with large operating voltage window that can develop rapidly a SEI layer to prevent plating of Li on a carbon anode and allows fast charging of the battery.\cite{Goodenough2010} Figure \ref{fig:electrolyte} schematically shows the electronic energy levels in electrode and electrolyte of a battery cell. If the anode electrochemical potential, $\mu_{A}$ is above the lowest unoccupied molecular orbital (LUMO) of the electrolyte, the electrolyte will get reduced, unless a passivating layer creates an energy barrier for transfer of electrons from the anode to the electrolyte LUMO. Similarly, if the cathode electrochemical potential $\mu_{C}$ is below the highest occupied molecular orbital (HOMO) of the electrolyte, the electrolyte will get oxidized, unless a passivating layer creates an energy barrier for transfer of electrons from the electrolyte HOMO to the cathode. Therefore, the electrochemical potentials $\mu_{A}$ and $\mu_{C}$ should lie within the the energy separation (E$_g$) between the LUMO and the HOMO of the electrolyte, which constrains the open-circuit voltage $V_{\rm oc}$ of a battery cell:

\begin{equation}
    eV_{\rm oc}=\mu_{A}-\mu_{C}\leq E_{g},
\end{equation}

where $e$ is the magnitude of the electron charge. %A passivating SEI layer at the electrode/electrolyte boundary can give a kinetic stability to a larger $V_{\rm oc}$ provided that $eV_{\rm oc}-E_{g}$ is small. 
The formation of the passivating SEI layer comes with the cost of irreversible capacity loss.

\begin{figure}
    \centering
    \includegraphics{figures/electrolyte.jpeg}
    \caption{Schematic open-circuit energy diagram of an aqueous electrolyte. $\Phi_{A}$ and $\Phi_{C}$ are the anode and cathode work functions. $E_{g}$ is the window of the electrolyte for thermodynamic stability. A $\mu_{A}>$ LUMO and/or a $\mu_{C}<$ HOMO requires a kinetic stability by the formation of SEI layer. Reprinted with permission from Ref.~\citenum{Goodenough2010}. Copyright 2010 American Chemical Society.}
    \label{fig:electrolyte}
\end{figure}

The gap $E_g$ for an aqueous electrolyte is $ \approx 1.3 $  eV which severely limits the open circcuit voltage $V_{oc}$. In order to obtain a higher open circuit voltage $V_{oc}$, non-aqueous electrolytes with larger $E_g$ seem promising candidates. Several lithium salts are soluble in some non-aqueous solvents and polymers, which has led to the development of modern Li-ion batteries.

Apart from larger thermodynamic stability window, another requirement is high ionic conductivity ($>10^{-4}$ S/cm) in the electrolyte and across the electrode-electrolyte interface for a high rate-capability. Figure \ref{fig:conductivity} shows the ionic conductivities of common non-aqueous electrolytes.\cite{Kamaya2011} These can be broadly classified into liquid and solid electrolytes. In the next subsections, we describe some of the liquid and solid electrolytes, their ionic diffusivity and stability.

\begin{figure}[htbp]
    \centering
    \includegraphics[scale=0.55]{figures/solid_electrolytes.pdf}
    \caption{Ion conductivity of several well-known solid lithium ion conductors, including glass and crystalline conductors. Reproduced from Ref.~\citenum{Zhang2018se_review} - Published by The Royal Society of Chemistry.}
    \label{fig:conductivity}
\end{figure}

%General picture and types of electrolytes. The importance of moving from liquid to solids and the challenges involved.
%General description, broad picture.
%Compatibility between electrolyte types and anode/cathode types. In terms of the cathodes, we discuss layered oxides (NMC), layered oxides (LiMn$_2$O$_4$), and polyanions (LiFePO$_4$), so if we can link electrolytes to these specific materials, it would prelude to the cathodes section. Similarly to the anodes section if possible?




\subsection{Liquid Electrolytes (Sam/Arihant)}
\label{sec:Liquid_electrolytes}
% \subsubsection{Introduction ()}
%Brief intro to liquid electrolytes, highlighting the importance of atomistic modelling. Introducing widely used liquid electrolytes, mixtures with solvents/common solvents. What materials/properties will be discussed here i.e. used as examples.

%Maybe something about impotance of electrolytes, like:
%Understanding electrolytes is very important in overall electric batteries modeling, as electrolytes may be ``bottle-necks'' in carrying high current density. may strongly affect degradation mechanisms, thermal effects, etc.

% The most widely used liquid electrolyte in Li-ion batteries is LiPF$_6$ in a solvent, which is typically a mix of two or more solvents, for example propylene carbonate (PC), ethylene carbonate (EC), ethyl methyl carbonate (EMC), dimethyl carbonate (DMC), in order to achieve competing objectives such as ability to dissolve high concentration of salt, low viscosity, high dielectric constant, at typical operational temperatures. Cyclic carbonates (EC, PC) have higher dielectric constant but also high viscosity, while ``linear'' carbonates (DMC, EMC) low viscosity but also low dielectric constant; for that reason often mixtures of solvents are used to optimise performance in a specific application are used.

% \begin{figure}
%     \centering
%     \includegraphics[scale=0.3]{figures/LE1.PNG}
%     \caption{Molecular structure for solvents: propylene carbonate (PC), ethylene carbonate (EC), dimethyl carbonate (DMC).}
%     \label{fig:LE1}
% \end{figure}

\subsubsection{An introduction to modelling liquid electrolytes}
The modelling of liquid electrolytes for conventional batteries is a broad and diverse field. Over the past 20-30 years atomistic modelling has helped shape the fundamental physics of liquids, determining a new physical basis and validating decades old pen and paper theories of concentrated electrolytes.The development of liquid electrolytes can be presented from an applications approach or a methods approach. Here, we focus on the development of liquid electrolyte models and the considerations needed when modelling these materials, before moving on to their applications in measuring different properties.

Atomistic modelling of liquid electrolytes can be broadly separated into \textit{ab initio} (first principles) and classical (potentials-based) Molecular Dynamics (MD) modelling (c.f. section~\ref{sec:molecular_dynamics}). These are complementary techniques which can be used to aid the other. For example, first principles calculations are able to provide information on the electron distribution which are required for parameterising the non-bonded components force fields used in classical MD. Classical MD can also be used to provide the starting conditions for DFT calculations. At their point of highest symbiosis \textit{ab initio} and classical methods are combined in quantum mechanics/molecular mechanics (QM/MM) studies, where the larger system is treated classically with a smaller sub-region being modelled using \textit{ab initio} methods. For example, a study by \citeauthor{Fujie_2018} used the ``Red Moon'' method to investigate the formation of the solid-electrolyte interface (SEI) at the metallic electrode\cite{Fujie_2018}.

In this section we first discuss the separate design and us of \textit{ab initio} and classical MD methods, followed by their application for investigating properties in the liquid electrolyte bulk. Finally, we discuss the application of atomistic methods for investigating the solid-electrolyte interface (SEI) from the perspective of the liquid electrolyte (complementary to the solid focused SEI discussion given in section~\ref{sec:SEI}).

\subsubsection{\textit{ab initio} modelling of liquid electrolytes}
\textit{ab initio} modelling using DFT (c.f. section~\ref{sec:dft}) provides a wide variety of functionals, each with different levels of described physics. A major limitation of this approach is the feasible time and length scales reachable due to high computational cost; generally restricting to a maximum of $\sim$ 30 nm$^3$ and tens to hundreds of pico-seconds. The limited time and length scales both introduce inaccuracies and irregularities in the calculations. Small length scales (box size) introduce issues regarding long to medium range correlations of atoms and molecules. As liquid do not exhibit long range order, the presence of periodic images that are located at exactly a cell width in all direction introduces an unphysical correlation. This is observed in the modelling of systematically disordered solids in smaller cells\cite{Morgan_2011}. For example, \citeauthor{Zhao_2020} recently revealed there is a distribution of different low-symmetry local motifs in cubic halide perovskites, such as tilting and rotations, which are only observed if you allow for a larger-than-minimal cell size \cite{Zhao_2020}. Beyond truncating the radial distribution function (RDF) to a shorter distance (i.e. half the shortest distance between periodic images) than is optimal, this effect will also introduce (normally small) inaccuracies in thermodynamic and dynamic quantities\cite{Binder2009book, yeh_system-size_2004,botan_diffusion_2015,horbach_finite_1996}. These inaccuracies are of a particular concern in liquid electrolytes as the electrostatic interactions between ions gives rise to longer range interactions \cite{coles_correlation_2020}. Use of a relatively short Debye length of concentrated electrolytes to counteract any correlations with periodic images is not plausible as the Debye H\"{u}ckel screening length is not relevant in these systems. This is because charge oscillatory screening with a decay length on the same order of magnitude as the standard box length is observed in these systems\cite{coles_correlation_2020}. The short time scale of \textit{ab initio} simulations can, particularly for more viscous liquids, lead to highly non-ergodic (fully-sampled) simulations. When snapshots throughout the whole trajectory are highly correlated\cite{frenkel_understanding_2002}, this can lead to problems for both dynamic and equilibrium studies. 

Neither the issue of time correlation or finite size effects has a significant detrimental to the results of \textit{ab initio} studies. However, in specific studies where they need to be avoided, or where a quantum description of a liquid electrolyte provides no significant advantage over a classical description, it is beneficial to turn towards far less computationally expensive (and thus larger and longer) potentials-based simulations.

\subsubsection{Classical modelling of liquid electrolytes}
Classical simulation of liquid electrolytes includes classical force field based molecular dynamics (c.f. section~\ref{sec:molecular_dynamics}) and the related field of classical Monte Carlo (c.f. section~\ref{sec:monte_carlo}). Classical MD, also known in solid-state communities as potentials-based MD, is a broad field which uses many different types of force fields for different studies. The development of force fields for ionic solids is described in section~\ref{sec:potential_fitting}, whereas here we evaluate the force fields used for liquid electrolytes and the considerations need to develop them. The focus here will be placed on the development of force fields for ionic liquid electrolytes, however, similar processes have taken place for organic solvents, which we will link to throughout the discussion.

Electrolyte solvents from water, to molecular solvents and ionic liquids, pose a challenge that is not normally present in the solid state, specifically the need to model covalent bonding. This is achieved by splitting the potential acting on each atom into bonding and non-bonding contributions. The non-bonding component accounts for the effects of electrostatics, dispersion, and degeneracy pressure; and the bonding component accounts for the effects covalent bonding. In classical modelling of liquid electrolytes we are mainly interested in the behaviour within the electrolyte's electrochemical window, therefore, the vast majority of classical studies model bonds with unbreakable, harmonic, potentials. There are four distinct types of bonded potential \cite{lindahl_gromacs_2021, frenkel_understanding_2002}: bonds, angles, dihedrals, and improper dihedrals. These can be traced back to the parameterisation of force fields, such as OPLSA-AA\cite{canongia_lopes_clp_2012,jorgensen_development_1996}, and are often parameterised from spectroscopic force constants. There are many ways of defining boned potential types, as described in the manuals of the popular Gromacs\cite{lindahl_gromacs_2021} and Lammps\cite{PLIMPTON19951} software, though their discussion is beyond the scope of this review. Atoms which are subject to a bonded potential are often wholly, or partially, excluded from non-bonded interactions as part of a force fields design. In larger molecules, intramolecular interactions beyond these exclusions will often play an important role in the description of a molecule.

Bonds can also be represented using Constraints, through algorithms such as LINCS\cite{hess_lincs_1997}, SHAKE\cite{ryckaert_numerical_1977}, and RATTLE\cite{andersen_rattle_1983}. These convert a molecule to a rigid body with fixed equilibrium bond lengths (or angles). This method gives greater computational efficiency by decreasing the degrees of freedom and allows for an initial simulation with an undefined force field. It should be noted that while constraining a C-H bond may have little effect, constraining other bonds or angles could readily affect structural and dynamic properties \cite{de_wijn_internal_2011, hess_lincs_1997}, and a constrained molecule is a single (non-spherical) statistical mechanical unit.

When developing force fields, generally, it is the non-bonded force field components, in particular the partial charges on atoms, which are more frequently varied. A common model for liquids electrolytes is the OPLS-AA force field developed by \citeauthor{jorgensen_development_1996} \cite{jorgensen_development_1996}. The non-bonded force field components consist of a Lennard-Jones potential, modelling the repulsive degeneracy pressure and the attractive dispersion, and a coulombic term, solved via Ewald summation \cite{ewald_berechnung_1921} or Ewald based grid method \cite{darden_particle_1993,deserno_how_1998,yeh_ewald_1999}.

The Canongia Lopes \& Padua (CP\&P) force field \cite{canongia_lopes_clp_2012, canongia_lopes_modeling_2004, canongia_lopes_molecular_2004, canongia_lopes_molecular_2006} was created to describe a wide range of ionic liquid cations and anions. It was originally based on the OPLS-AA force field for organic molecules (used for organic solvents in electrolytes), with non-bonded potentials and partial charges varied to model the ionic components. The charges on the individual molecules are obtained form first principles calculations (DFT), in this case by use of the charge mapping algorithm CHelpG\cite{canongia_lopes_clp_2012} (though other algorithms may also be used\cite{spackman_potential_1996,breneman_determining_1990,singh_approach_1984}). At the core of its design CL\&P had the principle of miscibility, meaning there is a separate force field for each cation and anion, with no change implemented for an ion on the basis of its counter-ion. CL\&P and OPLS-AA based force fields for ionic liquid and molecular lithium salt solutions are often employed with charge rescaling, equivalent to adjusting to relative permittivity of the media to a value other than one. The rescaling decreases the charge on ions and resolves the problem of OPLS-AA based force fields giving high viscosity for concentrated electrolytes, ionic liquids and their mixtures\cite{schroder_comparing_2012, schroder_polarizable_2020, shimizu_structural_2015}. Charge rescaling accounts for the effect of polarisability on the strength of electrostatic interactions between ions. However, other force fields have been defined to account directly for polarisability \cite{schroder_polarizable_2020}.

As described in section~\ref{sec:potential_fitting}, polarisability can be introduced to a force field using a type of core-shell model, also called the Drude oscillator model \cite{schroder_polarizable_2020, schroder_comparing_2012, lindahl_gromacs_2021}. The Drude oscillator model is computationally cheap and is core to the polarisable ionic liquid force field developed from CL\&P by \citeauthor{schroder_comparing_2012} \cite{schroder_comparing_2012}. A more advanced representation of polarisability can be provided by an intrinsically polarisable force fields, normally based on the Fumi-Tosi potential. Polarisability is incorporated within the non-bonded functional form of the non-bonded terms which give an intrinsic polarisability and has been used for molten salts \cite{madden_covalent_1996}, ionic liquids \cite{borodin_polarizable_2009, schroder_polarizable_2020}, and lithium salts in molecular solvents \cite{borodin_litfsi_2006,bedrov_molecular_2019,bedrov_influence_2010}. This provides the best description of polarisability in a classical force field, however, there is an associated higher computational cost. These force fields are also not available in many of the major molecular dynamics codes, and will often require a special code in order to be run (such as the metalwalls \cite{marin-lafleche_metalwalls_2020}).

The development of force fields for metal cations has seen an equal level of discussion and interest. When considering cations, such as Lithium or Magnesium, as shown by \citeauthor{mamatkulov_force_2013} \cite{mamatkulov_force_2013,mamatkulov_force_2018}, modelling these ions as small mildly dispersive non-polarisable spheres is a rather uncontroversial choice\cite{schroder_polarizable_2020}. In order to maintain compatibility with solvent force fields such as OPLS-AA and the simple point charge (SPC) water molecules these ions are frequently modeled as Lennard Jones spheres. In the Lennard Jones force fields of alkali and alkali earth metal cations a wide range of values of $\sigma$ (excluded volume) and $\epsilon$ (interaction strength) are used. This is because the basic energetics associated with one of these force fields can be recovered for many sigma values provided they are paired with a corresponding and correct epsilon value. The specific decision depends on the exact sort of properties that need to be accurately reproduced, such as the location and height of certain peaks within an RDF and the activity of solutions in a more concentrated regime\cite{mamatkulov_force_2013}. It is worth noting that many force fields used to modeled the electrolytes of specific interest to us here, were parameterised for aqueous solutions.

% \subsubsection{MD and analysis of liquid electrolytes}
%Here, we discuss the specific methods applied to model and analyse liquid electrolytes. In this section we will touch on some specific analyses and methods of particular utility to this specific field\footnote{A reader with more general interest is directed towards Frenkel and Schmit\cite{frenkel_understanding_2002}, while a reader with a specific interest in the physics of electrolyte solutions is directed toward Chapter 5 of Hansen and McDonnald}.
\subsubsection{Li-ion Diffusion}
%Diffusion
Diffusion (c.f. section~\ref{sec:diffusion}) plays a critical role in the operation of liquid electrolytes through its impact on conductivity. However, in liquid electrolytes its impact goes deeper as the dielectric constant of liquids consists of both dipolar and ionic contributions. These two contributions can be obtained by analysis of the dipole orientation and current auto-correlation functions using the Einstein-Helfand method. For example, \citeauthor{coles_correlation_2020} performed this analysis on four liquid electrolytes (three in aqueous solvent and one in a common organic solvent mixture): aqueous solutions of LiCl, NaI, and LiTFSI, as well as the same LiTFSI salt solvated in an equimolar mixture of DME and DOL \cite{coles_correlation_2020}. Here, it was shown that for polar solvents the dipolar contribution is nearly always dominant, with current acting as correction which could feasibly be neglected (particularly for more dilute systems). For ionic liquids, which contain ionic species that can exhibit a net dipole, such as bistriflimide. The effect of molecular ions having simultaneous charges and dipoles was explored by \citeauthor{schroder_collective_2011} who showed even more thorough treatment may be required to observed the impacts of their interplay\cite{schroder_dielectric_2009}.

Investigation of the diffusion of different ions subject to a field gives a sense of the diffusion rate of specific ions and also an idea of exchange rates of solvent molecules. As strongly coordinate solvents will have diffusion coefficients closer to the ions they are coordinated to than less strongly coordinating ligands \cite{shimizu_structural_2015, lesch_influence_2015, borodin_litfsi_2006,borodin_li_2006,borodin_li_2007}. Examples of this behaviour can be found in the studies of \citeauthor{borodin_li_2006} looking at diffusion in lithium solutions of both the common carbonate\cite{borodin_litfsi_2006} and elthylene glycol oligomer solvents\cite{borodin_li_2006}. The 2015 study of \citeauthor{shimizu_structural_2015} investigating a number of different lithium glyme solvate ionic liquids\cite{shimizu_structural_2015} and the study of \citeauthor{lesch_influence_2015} investigating lithium salts dissolved in aprotic ionic liquids.
In more complex solvents such as ionic liquids the nature of solvent plays a important role too, for instance \citeauthor{borodin_li_2007} showed the effect of flouriation of ionic liquid cations on diffusion behaviour\cite{borodin_li_2007}. This sort of study can be directly contrasted with pulsed field gradient nuclear magnetic radiation (NMR) experiments of the type normally used to study battery materials. This was done, for example, when \citeauthor{shimizu_structural_2015} studied lithium bistriflimide based solvate ionic liquid which had been proposed as a solvent for Lithium Sulfur batteries \cite{shimizu_structural_2015}.

\subsubsection{Bulk Structure and Landscaping}
%Bulk structure and landscaping(?)
For structural analysis of liquid electrolytes, analysis of the radial distribution function (RDF) is the mostly widely used approach. RFDs can be converted to structure factors by a simple Fourier transform into reciprocal space allowing for easy comparison with experimental structure factors\cite{shimizu_structural_2015, pethes_comparison_2017, hanke_intermolecular_2001,tsuzuki_molecular_2009}, subject to re-scaling for the specific intensities associated with different atoms. This method has been frequently used for a broad array of electrolytes, and has seen particular utility for ionic liquids, where the large inhomogenous ion surface can lead to complex patterns for which molecular dynamics can provide explanation. Modelling of this sort of behaviour has been preformed for aprotic\cite{Migliorati_2015} solvate ionic liquids\cite{shimizu_structural_2015},  imidazoliume salts\cite{hanke_intermolecular_2001}, lithium carbonate solutions\cite{Chaudhari_2018} and highly concentrated aqueous solvents\cite{pethes_comparison_2017}.

The physical relevance of RDFs does actually go further that this however, the RDF is closely related to potential of mean force acting on a particle. This describes the changing potential landscape acting between particles as they approach one another\cite{frenkel_understanding_2002}. As well as being generated from a RDF, the potential of mean force can be obtained by direct calculation by use of centre of mass pulling, umbrella sampling\cite{lindahl_gromacs_2021}, or running multiple calculations with ions frozen an exact distance apart from one and other. When modelling liquid electrolytes this method is also used to study the approach of ions to an electrode where the energetics associate with decoordination from the solvent and coordination to the electrode can be modelled. For example, \citeauthor{sergeev_electrodeelectrolyte_2017} looked at the approach of oxygen and lithium based species towards electrodes\cite{sergeev_electrodeelectrolyte_2017}.

\subsubsection{Solvation Energies}
%Solvation energies
Solvation energies in electrolytes have been widely studied and though research focus has been on aqueous solvation of biomolecules, these techniques can also be used to look at solvation of metal ions with organic solvents. Dependent on the exact thermodynamics of the system, the solvation energies of ions may be obtained by a number of methods.\citeauthor{Skarmoutsos_2015} used joint DFT and molecular dynamics methods to look at the solvation structures of lithium salts in ternary mixtures of different carbonate solvents. \cite{Skarmoutsos_2015} \citeauthor{Skarmoutsos_2015} showed that different solvents in the ternary mixture were found to dominate at different distances from a central lithium anion, with a particular preference for solvation of lithium by dimethyl-carbonate ions over propylene carbonate and ethylene carbonate being observed. \citeauthor{Takeuchi_2012} looked even deeper at the energetics behind the direct contact between cations and anions in solution\cite{Takeuchi_2012}.

These are just a few notable studies on solvation energies in liquid electrolytes. As is an exceptionally broad field, it can not be extensively covered within this review. A full theoretical description of solvation is given by \citeauthor{Lazaridis_1998} in Ref.~\citenum{Lazaridis_1998}.

\subsubsection{Solid-Electrolyte Interfaces (SEI)}
%Interfaces
In sections \ref{sec:anodes_surfaces_interfaces} and \ref{sec:cathode_interfaces} the interfaces between solids and liquids from the perspective of the solid have been discussed. However, the interface from the perspective of the liquid is also of interest. The structure of liquid electrolytes at metallic\cite{merlet_simulating_2013} and charged dielectric\cite{smith_electrostatic_2016} interfaces will normally extend nanometers away from the interface region.

Concentrated electrolytes and ionic liquids both adopt the characteristic overscreening structure at charged interfaces, including electrodes. This structure, comprising oscillations of charge decaying into the bulk, is commonly observed\cite{coles_nanostructure_2017,merlet_simulating_2013}. Modelling these systems requires an appropriated electrode model. While interesting information can be gained from simulating ions at an electrode with a fixed charge, for example in a high throughput study looking at structural changes with electrode surface charge\cite{coles_nanostructure_2017}, fixed potential boundary conditions will provide a more accurate description of the capacitance \cite{merlet_simulating_2013, scalfi_semiclassical_2020}, interfacial structuring of a liquid electrolyte \cite{coles_simulation_2019, vatamanu_ramifications_2017, li_capacitive_2018}, and the decoordination and dechelation dynamics of coordinated ions\cite{vatamanu_molecular_2009}. Though we note that in light of a recent study by \citeauthor{scalfi_semiclassical_2020} this field continues to evolve as more and more nuanced classical electrode models are employed, such as the Thomas-Fermi based model proposed by \citeauthor{scalfi_semiclassical_2020} \cite{scalfi_semiclassical_2020}.

A wide variety of different electrolytes have been studied using fixed potential electrolytes from ionic liquids to concentrated electrolyte. Both nanoporous \cite{merlet_highly_2013, merlet_molecular_2012, vatamanu_molecular_2009, vatamanu_ramifications_2017} and nanoscopically rough electrode surfaces have been heavily used\cite{vatamanu_influence_2011}. A specific example of interest is the work of \citeauthor{borodin_interfacial_2014}. The work of \citeauthor{Simoncelli_2018} looking at aqueous alkali earth electrolytes is of particular note here also, though its direct application is to Blue Energy Capacitors, this study showed the interplay between solvation and intercalation in nanoporous electrodes. A clear point that can be observed from study of such an electrode is the asymmetry that can be seen between even the simplest cations and anions, with sodium chloride ions having markedly different solvation numbers, which in turn affects electrode capacitance\cite{Simoncelli_2018}.  

Finally classical studies have looked at the structure of electrolytes at interfaces comprising SEI \cite{borodin_interfacial_2014}.

\subsubsection{Activity coefficients of electrolytes (Arihant)}
The activity coefficients describe the deviation of actual electrolytes from an ideal mixture of substances.\cite{Atkins2014} The activity coefficients of electrolyte can be calculated using DFT+P-BE simulations of solutes in electrolyte solutions as described in sec.\ref{sec:tf}. The experimental value of bulk permittivity of ethylene carbonate is ($\veps^\infty=90.7$)\cite{Hall2015} and the surface tension of EC is (0.0506~N m$^{-1}$)\cite{Naejus2002}. These values were used by \citeauthor{Dziedzic2020} to calculate the activity coefficient of LiPF$_6$ in ethylene carbonate (EC) solvent. \cite{Dziedzic2020} The solvent radius was set to $R^\textrm{solvent}_k= 10.5~a_0$ to approximate the size of an EC molecule, and the isovalue of solute electronic density ($\rho_{\textrm{e}}^\lambda$) is varied to match the experimental activity coefficients. A plot of the computed activity coefficients as a function of the square root of electrolyte concentration is given in Figure~\ref{fig:ac}, along with experimental values from Ref.~\citenum{Stewart2008}. Here, we see a good agreement for $\rho_{\textrm{e}}^\lambda=0.002~e/a_0^3$. Trends are also plotted from the linearised approximation of P-BE where the solvent radius is reduced to resemble the prediction for point charges from the Debye-H\"uckel theory.\cite{debye1923theory} The thermodynamic factor can be obtained from numerically differentiating these curves. This is a novel technique of calculating activity coefficients and thermodynamic factors from hybrid atomistic-continuum methods.

\begin{figure}
    \includegraphics[scale=0.8]{figures/lipf6.png}
    \caption{Mean activity coefficients for LiPF$_6$ in ethylene carbonate at $T=308$~K as a function of concentration and for different values of the atomic electronic density isovalue parameter which determines the extent of the accessibility function. Calculations with the linearised approximation to P-BE are also shown. Reprinted with permission from Ref.~\citenum{Dziedzic2020}. Copyright 2020 American Chemical Society.}
    \label{fig:ac}
\end{figure}
    
\subsubsection{Outlook and challenges (Arihant/Sam)}
% \begin{itemize}
%     \item discrepancy between atomistic/continuum definition of ``diffusion coefficient'' in terms of parameterising for continuum modelling.
%     \item limitation of liquid electrolytes
% \end{itemize}
%end: unchanged in 30 January 2021 update

\subsection{Solid Electrolytes (Lucy/Julian/Arihant/Rana)}
\label{sec:solid_electrolytes}

\subsubsection{Introduction (Rana/Lucy/Arihant)}
%Brief History of solid electrolytes, types of solid electrolytes, types of properties which are important for batteries. Then which we will focus on in this section sulfides (using LGPS and argyrodites), oxides (Examples), composites (example).
% Possibly we can select some of the below to mention. I think the Van de Ven review covers them all, but we can focus on a few.
% Oxides - NASICONs, LISICONS, perovskites (LLTO), Li garnets, nanocomposites (ranas)
% sulfides - glass ceramics, thio-LISICON (LGPS), argyrodites.
% Others - nitrides, oxynitrides (LiPON), antiperovskites

Solid electrolytes have attracted considerable attention as an alternative to highly-flammable liquid electrolytes, as they significantly improve device safety and have the potential to improve energy and power densities, while also reducing the cost of synthesis. \cite{janek_solid_2016, culver_designing_2018, famprikis_fundamentals_2019, goodenough_li-ion_2013, DIRICAN201927} An ideal solid electrolyte material should possess  high electronic resistance, high ionic conductivity, outstanding thermal stability, strong electrochemical stability, excellent mechanical strength, and reduced interfacial resistance. \cite{han2020recent, manthiram2017} There are three different categories of solid electrolytes which are used in rechargeable batteries \cite{DIRICAN201927}: (1) Inorganic ceramic electrolytes, (2) Organic polymer electrolytes, and (3) Composite electrolytes. 

Solid electrolytes were discovered by Michael Faraday in the early 1830s through research on the conduction properties of heated solid sliver sulfide (Ag$_{2}$S) and lead fluoride (PbF$_{2}$) \cite{Faraday1833}. 
The use of a ceramic-based $\beta$-alumina (Na$_{2}$O$\cdot$11Al$_{2}$O$_{3}$) in high-temperature sodium-sulfur batteries in the 1960s was considered as a milestone in the development of batteries enabled by solid electrolytes \cite{armand2008building}. In the 1980s the Zeolite Battery Research Africa (ZEBRA) group developed the ``ZEBRA'' batteries using Na$_{2}$O$\cdot$11Al$_{2}$O$_{3}$ as the solid electrolyte. \cite{ZEBRA}
So far, the high-temperature sodium–sulfur battery has been commercialised in Japan \cite{oshima2004}, whereas the ZEBRA battery is currently being developed by the General Electric Corporation in the United States. \cite{capasso2014} 

In 1990, the Oak Ridge National Laboratory synthesised a lithium phosphorus oxynitride (LiPON) material \cite{dudney1992,bates1992}, which opened up the use of inorganic solid-state electrolytes in lithium-ion batteries. Since then, a huge number of inorganic lithium-ion conductive ceramic materials have been developed, including perovskite-type \cite{inaguma1993}, garnet-type oxides \cite{kasper1969,mazza1988}, garnet-type sulfides \cite{kennedy1986}, lithium super ionic conductor (LISICON) \cite{ivanov1988}, sodium super ionic conductor (NASICON)-like materials \cite{lang2015}, lithium-argyrodite materials, \cite{deklerk2016} and Li-rich anti-perovskites. \cite{dawson2018elucidating,ahiavi2020mechanochemical}

Despite recent advancement in research activities on crystalline inorganic electrolytes, they are still brittle and therefore difficult to fit into different battery shapes. Solid-state polymer electrolytes (SPEs), due to their high flexibility, can fit in any battery shape and present improved safety and stability features compared to crystalline inorganic electrolytes. \cite{DIRICAN201927} Since 1980, various high-molecular weight dielectric polymer hosts were investigated as polymer electrolytes with high conductivities for lithium batteries, such as poly(ethylene oxide) (PEO) \cite{fenton1973}, polyacrylonitrile (PAN) \cite{abraham1990,dautzenberg1994}, poly(vinylidene fluoride) (PVDF), \cite{arcella1999,kataoka2000,li2016} poly(methyl methacrylate) (PMMA) \cite{appetecchi1995,bohnke1993} and poly(vinylidene fluoride-hexa-fluoropropylene) (PVDF-HFP) \cite{abbrent2001,park2008,yang2014}.

The ionic conductivities of most polymer electrolytes are significantly lower than those of both oxide solid electrolytes and liquid electrolytes. \cite{zhou2016} A possible solution to this limitation is to create composites by integrating nanoscale highly-conductive inorganic particulate fillers into the polymer electrolyte material. \cite{DIRICAN201927} This enhances the ionic conductivity and also improves the mechanical strength and stability of the solid-state polymer electrolytes, including the interfacial stability. \cite{D0SC03121F} Here, heterogeneous doping increases the ionic conductivity as a result of increasing interfacial regions between an inert solid phase, such as silica or alumina or boron oxide particles and an electrolyte. \cite{uvarov2011} A wide range of inorganic solid composite electrolytes have previously been studied, based on oxides (Li$_{2}$O:Al$_{2}$O$_{3}$ \cite{B300908D}, Li$_{2}$O:B$_{2}$O$_{3}$ \cite{Heitjans_2003,Indris2000,Indris2002}), hydrides (LiBH$_{4}$:SiO$_{2}$ \cite{blanchard2015}), halides (LiI:Al$_{2}$O$_{3}$ \cite{liang1973}, LiI:SiO$_{2}$ \cite{phipps1983}, LiF:Al$_{2}$O$_{3}$ \cite{uvarov1992}), and sulfides (Li$_{2}$S:SiS$_{2}$ \cite{pradel1986}). 

Over the last decade, a limited number of candidates with high ionic conductivities ($>$1 mS cm$^{-1}$) have arisen as potential competitors to liquid electrolytes.\cite{kanno_synthesis_2000, murayama_synthesis_2002, murayama_material_2004, minafra_influence_2019,bron_li_2013,whiteley_empowering_2014,huang_superionic_2019,yamane_crystal_2007,homma_crystal_2011} Figure~\ref{fig:conductivity} presents a plot by \citeauthor{Kamaya2011} with the ionic conductivities of most currently known solid electrolytes.\cite{Kamaya2011}

In this section, we review atomistic modelling investigations into the structure-property relationships in selected solid-state electrolytes; Li$_{10}$GeP$_2$S$_{12}$ (LGPS), lithium argyrodites, and Li$_7$La$_3$Zr$_2$O$_{12}$ (LLZO) which belong to the inorganic solid ceramic electrolyte type and Li$_{2}$O:B$_{2}$O$_{3}$ materials which belong to the oxide based solid composite type. A particular focus is given on the ion transport mechanism in those materials, which is important for reaching high conductivities, a key property of battery materials. Finally, we take a more detailed look at the interface of solid electrolytes with the electrodes, and discuss the challenges and outlook for future atomistic modelling investigations.

\subsubsection{Sulfides (Arihant/Lucy)}
There are a substantial number of computational studies of sulfides which largely related to a recent increase in newly discovered crystalline sulfide superionic conductors. Sulfides also tend to have comparatively lower intrinsic electrochemical and chemical stability, which has stimulated interest in understanding the interfacial interactions within batteries. \cite{Xiao2020interfacerev} The sulfide group encompasses a range of sulfide based solid electrolytes, including, glass ceramics \cite{minami2006recent}, argyrodites \cite{bai2020research}, and thio-LISICONs \cite{minafra2020two}. Some of the most promising SEs to emerge in recent years include the Li$_{10}$GeP$_2$S$_{12}$ (LGPS) \cite{Bhandari2016,Kamaya2011,Mo2012} and the Li-argyrodite (Li$_6$PS$_{5}X$, $X$=Cl,Br,I) \cite{kraft2018,deiseroth_li6ps5x_2008,deklerk2016,kraft2017,minafra2018,adeli2019} families of superionic conductors.

\textbf{LGPS}
A study by \citeauthor{Kamaya2011} reports Li$_{10}$GeP$_2$S$_{12}$ (LGPS) can reach high room temperature ionic conductivities of 12 mS cm$^{-1}$, comparative to commercial liquid electrolytes.\cite{Kamaya2011} The authors also determined diffusion in LGPS is anisotropic, where $c$ directional motion is predominant over the $ab$ plane, with an overall energy barrier for Li diffusion being 0.24 eV, with later reports measuring 0.22 eV.\cite{Kuhn2013b} Using \textit{ab initio} MD \citeauthor{Mo2012} later determined the average direction energy barriers of 0.17 eV along the $c$ channel and 0.28 eV in cross channel ($ab$ plane), \cite{Mo2012} with \citeauthor{Xu2012one} showing the Li migration mechanism is through cooperative motion instead of the initially determined single-hop mechanism.\cite{Xu2012one} More recently, \citeauthor{Adams2012} predicted the presence of additional Li sites using MD, which would allow diffusion along the $ab$ plane.\cite{Adams2012} These sites could change not only the Li occupancies in the $c$ channel, but also provide a diffusion mechanism involving $ab$ plane, opening up the possibility of cross-channel diffusion. The presence of these additional sites were later confirmed experimentally using single crystal XRD.\cite{Kuhn2013a}

More recently, \citeauthor{Bhandari2016} also investigated the lithium diffusion dimensionality in LGPS by performing a first-principles DFT study of the lithium diffusion energy barrier using the nudged elastic band (NEB) method.\cite{Bhandari2016} In this study the authors took into account the fractional occupancies leading to variable $c$ channel Li populations, variable chemical environments surrounding Li, and all possible migration mechanisms. The authors found that lithium diffusion is neither purely $c$ directional nor purely along the $ab$ plane, but there exists a correlated mechanism of motion along $c-ab$ which critically controls the degree of anisotropy of Li diffusion in LGPS. The energy barriers for different mechanisms of Li-diffusion, shown in Figure~\ref{fig:lgps}, suggest correlated hopping has the lowest energy barrier. %This correlated mechanism of Li-diffusion could not be deciphered from the experimental data or MD data alone, as these methods can only isolate diffusivities along particular directions. 
\citeauthor{Bhandari2016} further performed a statistical average of all diffusion energy barriers, taking into account the formation energy of various Li configurations and predicted an overall energy barrier of 239 meV, which is in close agreement with experiments.\cite{Kamaya2011} Thus, the first-principles approach not only explained the overall diffusivities and energy barriers, but also gave insight into the underlying mechanism behind the fast Li diffusion in LGPS and resolved the discrepancy about the anisotropy of Li diffusion in this compound.

\begin{figure}
    \centering
    \includegraphics[scale=1.2]{figures/lgps.png}
    \caption{Energy barrier for Li-ion diffusion in LGPS solid electrolyte calculated using NEB method. Reprinted with permission from Ref.~\citenum{Bhandari2016}. Copyright 2016 American Chemical Society.}
    \label{fig:lgps}
\end{figure}

\textbf{Lithium argyodites}, Li$_6$PS$_{5}X$ ($X$= Cl,Br,I) can reportedly reach ionic conductivities of up to $10^{-2}$ Scm$^{-1}$. \cite{deiseroth_li6ps5x_2008} While Li$_6$PS$_{5}$Cl and Li$_6$PS$_{5}$Br exhibit high ionic conductivities of $10^{-3}$ Scm$^{-1}$ at room temperature, Li$_6$PS$_{5}$I has considerably lower conductivitives of $10^{-6}$ Scm$^{-1}$.The three orders of magnitude difference is surprising as the identical crystal structures suggest the same Li diffusion pathways exist in all systems. Another intriguing aspect is that the conductivity trend runs counter to other families of SEs, such as LGPS, where larger, more polarisable and less electronegative anions are linked with increased ionic conductivites. \cite{bachman2016inorganic}

Understanding which properties and mechanisms influence the conductivity is essential to obtaining higher ionic conductivities and improving battery performance. Material stoichiometry, anion/cation disorder, and doping, have all been shown to influence conductivities. The effect of these can be difficult to deconvolve in many materials as they are intrinsically coupled in experimental systems. This is where computational analysis can provide vital insight, allowing deconvolution of coupled properties and the roles they play in the diffusion process.

A particularly interesting aspect of the Li-argyrodites is the diffusion topology, which comprises of interconnected Li$_6$S cages, with anions arranged at 4a, 4c, and 16e Wyckoff positions and Li arranged over type 1, 2, 4, and 5 tetrahedra.\cite{kuhs1979} Lithium mainly occupies type 5 tetrahedral sites in $x$(Li)=6 argyrodites, with occupation of non-type 5 sites only recently observed experimentally. \cite{ohno2019further,gautamengineering} Computational studies however have previously predicted occupation of non-type 5 sites, showing lithium distributed over tetrahedral types 5, 2, and 4. \cite{deiseroth_li6ps5x_2008, Minafra2020, morgan2020mechanistic}

Li hopping within these cages, while effectively barrier-less, does not contribute to long-range diffusion. In fact, a combination of inter-cage and intra-cage hopping is needed, with occupation of non-type 5 sites and transitions between all adjacent site types to achieve long-range diffusion. This is shown schematically in Figure~\ref{fig:diffusion_pathways} showing the connectivity between the Li tetrahedral sites. AIMD simulations have shown that cation and anion substitution \cite{ohno2019further,deklerk2016}, anion site disorder \cite{gautamengineering,morgan2020mechanistic}, and lithium concentration \cite{Deng2017, yu_superionic_2020, Feng_2020}, all influence the ionic conductivity.

\begin{figure}
    \centering
    \includegraphics[scale=0.65]{figures/diffusion_pathways.pdf}
    \caption{(a) Possible Li diffusion pathways in Li-argyrodites involving types 2, 4, and 5 tetrahedra for long range diffusion. Reprinted with permission from Ref.~\citenum{morgan2020mechanistic}. Copyright 2020 American Chemical Society.}
    \label{fig:diffusion_pathways}
\end{figure}

The influence of anion substituent concentration on conductivity is currently uncertain, with research by \citeauthor{deklerk2016} determining excess Cl in Li$_5$PS$_4$Cl$_2$ resulting in similar conductivities to Li$_6$PS$_5$Cl \cite{deklerk2016}, in contrast to research by \citeauthor{yu2019tailoring} and \citeauthor{Feng_2020} concluding excess Cl improved Li conductivity. \citeauthor{yu2019tailoring} determined the highest conductivity was produced by Li$_{5.7}$PS$_{4.7}$Cl$_{1.3}$ (6.4 mS cm$^{-1}$), \cite{yu2019tailoring,yu_superionic_2020} while \citeauthor{Feng_2020} determined this to be Li$_{5.3}$PS$_{4.3}$Cl$_{1.7}$ (17 mS cm$^{-1}$). \cite{Feng_2020} \citeauthor{Feng_2020}, however, presented alternative, or coupled, reasoning for this increased conductivity. Drawing from previous studies \cite{adeli2019,zhou_solvent-engineered_2019} they propose the increased Cl content amplified the anion disorder in the system, which is the underpinning cause of the higher conductivities.

\subsubsection{Oxides (Julian/Rana)}
\label{sec:se_oxides}
\textbf{LLZO} Cubic Li$_7$La$_3$Zr$_2$O$_{12}$ (c-LLZO) has a high Li-ion conductivity of 10$^{-4}$ Scm$^{-1}$\cite{Murugan2007}, a high shear modulus of 59 GPa\cite{Ni2012}, and the largest thermodynamic stability window against lithium metal\cite{Zhu2016, Binninger2020} of current solid electrolyte materials. However, at low temperatures <150 $^\circ$C c-LLZO is not stable and transitions to the less conductive tetragonal LLZO (t-LLZO) phase.\cite{Geiger2011} Attempts have been made to retain the more desirable c-LLZO by Al doping on lithium sites, with some success.\cite{Geiger2011, Rangasamy2012}

Lithium dendrite growth has shown to be a problem in solid-electrolytes. For LLZO, dendrite growth has caused short circuits in the cells after relatively short periods\cite{Ren2015,Sudo2014}. \citeauthor{Cheng2017} observed this growth directly and found the process occurs mostly through grain boundaries.\cite{Cheng2017} Recently, \citeauthor{Kim2020} confirmed these observations and is currently investigating the use of an interlayer buffer to restrict Li propagation through grain boundaries.\cite{Kim2020}

There has been a wide effort to understanding dendrite formation through modelling\cite{Canepa2018, Tian2018, Gao2020}. For example, \citeauthor{Tian2018} used DFT (c.f. section \ref{sec:dft}) to investigate dendrite growth through analysis of bulk/slab c-LLZO and t-LLZO energy surfaces via the total density of states (TDOS)  \cite{Tian2018}.The authors found that t-LLZO forms at the surface of bulk c-LLZO, even with Al-doping\cite{Ma2016, Rettenwander2018}, and extra states appear in the band gap for the slab structures, potentially allowing electrons to be trapped on the surface of LLZO. Electrons localised primarily around Li$^+$ and La$^{3+}$ ions on the surface leading to the nucleation of lithium metal, which can result in growth through grain boundaries and pores in the LLZO eventually forming dendrites\cite{Ren2015}, Figure~\ref{fig:tian2020}. This analysis was also conducted on Li$_2$PO$_2$N (LiPON), where no electron trapping was found to occur, indicating the material could be a suitable coating to prevent dendrite and t-LLZO formation (c.f. section \ref{sec:interface_stability}).

\begin{figure}[H]
    \centering
    \includegraphics{figures/tian_grain_growth.png}
    \caption{Schematic showing Li metal formation (blue) along grain boundaries and pores due to electron accumulation (red) combining with Li$^+$ as they move through the electrolyte. Reprinted from Ref.~\citenum{Tian2018}, Copyright 2018, with permission from Elsevier.}
    \label{fig:tian2020}
\end{figure}

In contrast, \citeauthor{Gao2020} attributes the dendrite growth mechanism to the under-coordination of Zr present on some of the stable interfaces of LLZO with Li,\cite{Gao2020} leading to inhomogenous Li depletion which has been linked to Li metal deposition and dendrite formation.\cite{Tsai2016} The different attributed mechanisms could stem from the choice of surface. \citeauthor{Tian2018} used Li and La rich surfaces, which were determined to be more stable by \citeauthor{Thompson2017}, who used DFT to investigate 6 different LLZO slabs for the (100) and (110) planes.\cite{Thompson2017} In contrast, \citeauthor{Gao2020} drew upon results presented in several methods\cite{Thompson2017, Canepa2018, Yu2016a} and performed DFT calculations on a wider range of surfaces, finding (100) and (001) surfaces the most stable. The findings of these studies agree that Li and La rich surfaces are the most stable. However, \citeauthor{Gao2020} calculated the interface formation energies of the Li-LLZO interfaces using the CALYPSO interface structure prediction method\cite{Wang2012, Gao2019}. This method determined the Zr-rich surfaces as the most stable. Experimental observations corroborate these findings, also determining Zr to cause interfacial degradation\cite{Zhu2019}.

Experimental measurements have suggested a non-uniform distribution of current on the surface causing Li reduction as an alternative cause of dendrite growth.\cite{Han2019_dendrite, Aguesse2017} \citeauthor{squires_2020} used DFT to model the electronic conductivity in LLZO to probe the importance of surface current to dendrite formation.\cite{squires_2020} The authors determined that at room temperature, bulk c-LLZO was found to have negligible electron/electron-hole concentrations, indicating that bulk defects are not a significant factor in dendrite growth. However, these models did not count for other forms of defects such as grain boundary and surface effects. 

Understanding Li-ion migration is key to improving battery conductivity. \citeauthor{Xu2012} analysed the Li-ion migration path through LLZO using DFT with the nudged elastic band (NEB) method (c.f. section \ref{sec:methods_neb}).\cite{Xu2012} Two migration paths were observed, depending on Li concentration. Low Li$_x$ (Li$_5$La$_3$Zr$_2$O$_{12}$) led to a higher energy, single hop, migration path, where higher Li$_x$ (Li$_7$La$_3$Zr$_2$O$_{12}$), led to a lower energy, two hop, migration path. Using potentials-based MD (c.f. section \ref{sec:molecular_dynamics}) \citeauthor{Burbano2016} further investigated the Li-ion transport mechanisms by comparing ionic conductivity in t-LLZO and c-LLZO.\cite{Burbano2016} The authors found the longer time scale of potentials-based MD allowed the observation of a large sample of diffusion events in both LLZO structural forms. Diffusion events in t-LLZO were less common and mostly involved exactly 8 Li ions, which corresponds to the cyclic movement of Li ions around the 12 octahedral and tetrahedral ring sites in t-LLZO. This cyclic mechanism results in no net long-ranged diffusion of Li and hampers the ability of t-LLZO to conduct ions. AIMD (c.f. section \ref{sec:molecular_dynamics}) investigations of the transport mechanism in LLZO have been conducted, however, the shorter time-scale led to some key disagreements about the transport mechanism in c-LLZO.\cite{Meier2014, Jalem2013, Burbano2016}

DFT calculations have determined Al doping reduces the energy barrier for Li-ions to move between octahedral and tetrahedral sites, increasing ionic conductivity.\cite{Rettenwander2014, Rettenwander2016} More recently, using potentials-based MD, \citeauthor{Bonilla2019} supports these findings, determining increased conductivity in -LLZO was due to the Al forcing Li ions into previously inaccessible tetrahedral sites.\cite{Bonilla2019} The authors also found Al doping in c-LLZO led to a slight decrease in conductivity. They attributed this to the tendency for Al to ``trap'' Li ions close to the dopant.

\textbf{Oxide Nanocomposites} Due to attractive mechanical, electrical, optical, and magnetic properties nanocomposite oxide materials represent a new generation of advanced materials. \cite{uvarov2011,Heitjans_2003} They often show enhanced conductivity compared to the single-phase ceramic oxides which makes them suitable candidates as electrolytes for future solid state batteries. For example, Li$_2$O:B$_2$O$_3$ \cite{Heitjans_2003,Indris2000,Indris2002} and Li$_2$O:Al$_2$O$_3$ nanocomposites \cite{B300908D} have higher ionic conductivities than nanocrystalline Li$_2$O, although B$_2$O$_3$ and Al$_2$O$_3$ are insulators. The ionic conductivity shows a maximum at about 50 \% of B$_2$O$_3$/Al$_2$O$_3$ content. This surprising behaviour was attributed to the increased fraction of structurally disordered interfacial regions and the enhanced surface area of the nanosized particles \cite{Heitjans_2003}. The oxide nanocomposites contain three types of interfaces, as presented in Fig.~\ref{fig:LBO} (a): interfaces between the ionic conductor grains (green lines), between the insulator grains (black lines), and between the ionic conductor and the insulator grains (red lines). The latter can lead to surprising effects in the conductivity of composite materials. In this case, the highly conducting interface region can act as a bridge between two Li$_2$O grains not in direct contact with each other, opening up additional paths for Li ions. The conductivity enhancement in the interfacial regions may have different origins, e.g. the formation of space charge layers, an enhanced concentration of dislocations, or defects or the formation of new phases.

\begin{figure}
    \centering
    \includegraphics[scale=1]{figures/Islam-Fig-Li2O-B2O3.png}
    \caption{(a) Schematic diagram of Li$_2$O and B$_2$O$_3$ interface (b) Atomistic model of Li$_2$O:B$_2$O$_3$ nanocomposite. Reproduced with permission from Ref.~\citenum{Rana-JPCM-2012} Copyright IOP Publishing. All rights reserved.}
    \label{fig:LBO}
\end{figure}

\citeauthor{Rana-PRL-2007} studied the interface of Li$_2$O:B$_2$O$_3$ nanocomposite by modelling a combination of two favorable surfaces of Li$_2$O and B$_2$O$_3$ using HF/DFT Hybrid approach. \cite{Rana-PRL-2007,Rana-JPCM-2012} After full structural optimisation, it was observed that Li--O bonds are weakened and simultaneously B--O bonds are formed at the boundary between the two surfaces, Fig. \ref{fig:LBO} (b). An oxygen atom from the Li$_2$O surface (marked by a green circle) is pulled from the surface layer towards a neighboring boron atom of the B$_2$O$_3$ surface. This preference of oxygen bonding with B (or Al in Li$_2$O:Al$_2$O$_3$) plays a key role in generating low-coordinated Li. As a consequence of this dislocation, the coordination of a Li atom in the second layer is reduced from four to three. 

The defect properties were investigated in the interface region. It was observed that the removal of surface oxygen from Li$_2$O is responsible for the increased vacancy defect concentration in Li$_2$O:B$_2$O$_3$ (or Li$_2$O:Al$_2$O$_3$) nanocomposite materials. Therefore the nanocomposites of ionic compounds (containing weakly bound and therefore mobile cations) with highly covalent compounds (with strong metal- or nonmetal-oxygen bonds) are in general promising candidates for high ionic conductivity. The model calculations showed that the most likely mechanism for Li$^+$ migration was in a zigzag pathway rather than in a straight line along a direction parallel to the interface plane. 

The average calculated activation energy for Li$^+$ migration in the Li$_2$O:B$_2$O$_3$ interface (0.28 eV) \cite{Rana-PRL-2007,Rana-JPCM-2012} is similar to the experimental values of bulk Li$_2$O (0.31 eV) \cite{Heitjans_2003}, Li$_2$O:B$_2$O$_3$ ($0.34 \pm 0.04$ eV) \cite {Indris2002} and Li$_2$O:Al$_2$O$_3$ ($0.30 \pm 0.02$ eV) \cite{B300908D} nanocomposites. According to the defect formation energies, the interface region of Li$_2$O:B$_2$O$_3$ nanocomposites contains higher concentrations of both Li vacancies and Frenkel defects than bulk Li$_2$O and the Li$_2$O surfaces. \cite{Rana-PRL-2007,Rana-JPCM-2012} Therefore the experimentally observed enhanced Li mobility in the Li$_2$O:B$_2$O$_3$ interface region is thermodynamically and not kinetically controlled. Models as proposed in that study allowed a direct simulation of the defect formation and ion mobility at atomic scale without any experimental input. They provided a deep insight into the local bonding situation at the interface of oxide nanocomposites which was difficult to obtain from experiments.

\subsubsection{Interface stability (Julian)}
\label{sec:interface_stability}
Experimental investigations of solid electrolyte interfaces are often challenging, making atomistic modelling a vital tool.\cite{Xu2018exp} It has been widely reported in both experiment\cite{Liu2013, Han2015} and theory\cite{Mo2012} that certain solid electrolytes have an electrochemical stability window against a Li anode between 0--5 V.\cite{Kamaya2011, Thangadurai2005, Liu2013} \citeauthor{Mo2012} reported a 3.6 eV band gap from a DFT calculation (c.f. section~\ref{sec:dft}) for LGPS,\cite{Mo2012} attributing the higher stability observed to the passivisation phenomenon forming an interphase layer at the interface of the anode-electrolyte.\cite{Kobayashi2008} More recent work by \citeauthor{Zhu2015} has questioned this high stability window, using DFT to demonstrate that the stability windows, particularly of sulfides, are far smaller than originally thought (Figure~\ref{fig:se_stab}).\cite{Zhu2015} Reducing the thermodynamic window increases the importance on the interphase layer formation. \citeauthor{Zhu2015} determined a range of solid electrolytes are unstable with respect to Li metal at low and high voltages with the exception of LLZO, which appears to be kinetically stabilised at low voltages due to an unfavourable reduction energy of -0.02 eV per atom. Any potential outside of the thermodynamic stability window results in decomposition into lithium binary compounds. This is problematic for germanium and titanium containing compounds as they form electronically conductive alloys.\cite{Zhu2015} This renders the proposed passivation process impossible\cite{Mo2012, Zhu2015} as this degradation would be sustained throughout the bulk cycling severely limiting the efficacy of these materials as electrolytes. Such degradation can also increase interfacial resistance\cite{Takada2008, Sakuda2010}. Other solid electrolytes face different problems. As explained in sec. \ref{sec:se_oxides}, LLZO forms the far less ionically conductive tetragonal LLZO at the surface. The Li-LiPON and Li-argyordites interfaces were reported to degrade favourably, forming an ionically conductive and electronically insulative interphase consisting of Li$_2$O, Li$_2$S, Li$_3$P, Li$_3$N, and LiI.\cite{Zhu2015} 

\begin{figure}[H]
    \centering
    \includegraphics[scale=0.45]{figures/SE_voltage_stability.png}
    \caption{A comparison of the voltage stability windows for a selection of solid electrolytes (green) and the binary compounds that often form upon decomposition of the solid electrolyte (orange). The dashed line represents the oxidation potential to fully delithiate the material. Reprinted with permission from Ref.~\citenum{Zhu2015}. Copyright 2015 American Chemical Society.}
    \label{fig:se_stab}
\end{figure}

Further study by \citeauthor{Zhu2016} sought to investigate the mechanism behind the degradation/instability at the surface.\cite{Zhu2016} In order to probe this the authors looked at several solid electrolytes (LGPS, LLZO, LiPON, NAISICON-type, LLTO) and calculated the chemical stability, electrochemical stability, and the equilibrium conditions at the interfaces. Examining the cathode-electrolyte interface, using lithium cobalt oxide (LCO), a similar pattern emerged. Oxides were far more stable than their sulfide counterparts. However, LLTO and LATP had the best electrochemical stability against LCO.

Studies looking into the interfacial resistance have been conducted,\cite{Tateyama2019, Okuno2020, Sharafi2017, Jiang2019} with the main source of resistance attributed o the electric double layer, which in liquid electrolytes consists of a capacitance and diffusion layer.\cite{Tateyama2019} The CALYPSO method was used to find low-energy surfaces\cite{Gao2019} to probe the interface, as done with liquid electrolytes (c.f. section.~\ref{sec:Liquid_electrolytes}). The lithium chemical potential of these stable interfaces in the Helmholtz layer, corresponding to the negative of the Li ion vacancy formation energy, was determined. These energies correspond to where the lithium will move from the electrode to the electrolyte first. These sites depleting first upon charging can be a source of interfacial resistance. \citeauthor{Okuno2020} use DFT calculations to compare the interfacial resistances of sulfide and oxide based solid electrolytes with LCO cathodes.\cite{Okuno2020} The Li vacancy formation energy at various interfaces and ion exchange across the interface were calculated. It was found that sulfide based electrolytes had a higher interfacial resistance due to the presence of more sites with a low vacancy formation energy on the surface. The authors also found the interfacial resistance to be dependent on the orientation of the crystals at the interface. The cause of interfacial resistance can also be attributed to preparation of solid electrolyte surfaces.

A study by \citeauthor{Lepley2015} used DFT to investigate the interface energies between the Li electrode and the compounds that make up the interphase layer of the electrolyte.\cite{Lepley2015} They defined the interface energy as:

\begin{equation}
    \gamma_{ab}(\Omega)=\frac{E_{ab}(\Omega,A,n_a,n_b)-n_aE_a-n_bE_b}{A},
\end{equation}

where $\Omega$ is the interface configuration of atoms, $E_{ab}$ is the energy of the complete system, $E_x$ is the bulk energy per for formula unit and $A$ is the surface energy. Because the interface energy is intensive, calculating larger systems will give a converging value for $\gamma_{ab}$,

\begin{equation}
    \lim_{\Omega_s \rightarrow \Omega} \left[\gamma_{ab}(\Omega_s)\right]=\gamma_{ab}(\Omega),
\end{equation}

where, $\Omega_s$ is the atomic configuration in a sample of the interface volume. Because the exact matching of lattice constants between interfaces is unlikely, a semi-coherent interface is considered, meaning lattice strain needed to be taken into accounted. Using the lowest overall lattice energy structure and explicitly accounting for the lattice strain, the most probable interfaces could be found. The Li/Li$_3$PO$_4$, Li/Li$_2$O and Li/Li$_2$S interfaces were found to be stable and the Li/Li$_3$PS$_4$ interface unstable.\cite{Lepley2015}

In response to the apparent poor stability of most solid electrolytes many studies have attempted to simulate the effect of coating the electrolyte with an oxide layer\cite{Zhang2020directvis, Xiao2019coat, Tian2018}. As discussed in section~\ref{sec:se_oxides}, \citeauthor{Tian2018} identified LiPON as a suitable coating material for LLZO by comparing the bulk and surface density of states\cite{Tian2018}. The authors found no extra states on the surface structure, so concluded that no electron trapping would occur (the primary mechanism that they attributed to dendrite formation). Recently, \citeauthor{Sang2020} proposed an artificial interphase layer between the Li anode and the solid electrolyte composed of a Li$_{3a_b}$N$_{a}X_{b}$ compound, where $X$ is a halide.\cite{Sang2020} 
This material was investigated computationally by screening stable and metastable structures using the USPEX structure prediction software.\cite{Glass2006, Oganov2006} The dynamic stability of the stable structures was found by analysing the phonon frequency spectrum by using \textsc{Phono3py} \cite{Parlinski1997, Togo2008,togo_distributions_2015}. The temperature-dependant ionic transport properties were found using \textit{ab initio} molecular dynamics (AIMD). (c.f. section~\ref{sec:molecular_dynamics})

Phase diagrams for various atomic configurations were then constructed through Alloy-Theoretic Automated Toolkit (ATAT) which uses cluster expansion (c.f. section~\ref{sec:cluster_expansion}).\cite{Hart2008, VandeWalle2002} Through these various computational techniques \citeauthor{Sang2020} found that Li$_6$NCl$_3$ to have the most favourable properties for use with sulfide-based solid electrolytes such as LGPS.\cite{Sang2020}

\subsubsection{Outlook and challenges (Julian/Lucy)}
\label{sec:outlook_electrolytes}
The drive for the development of commercialised all-solid-state batteries has been intense, with the electric vehicle industry being at the forefront of promoting this.\cite{Woods_2021} Although solid-state batteries can offer high gravimetric energy density (250 Wh kg$^{-1}$) and volumetric energy density (700 Wh L$^{-1}$), along with improved safety over convesntional liquid electrolytes, the slow kinetics can impair the fast discharge and charge performance. With solid electrolytes intended to replace both the separator and liquid electrolyte in conventional LiBs, \cite{schnell2020solid} there are still multiple challenges which need to be overcome for this to be viable. In recent years there have been breakthroughs in the discovery of new solid electrolytes, such as Li$_{9.54}$Si$_{1.74}$P$_{1.44}$S$_{11.7}$Cl$_{0.3}$, \cite{kato2016high} which exhibit ionic conductivity competitive with that of organic liquid electrolytes. The improved performance of these materials is enabled by interfacial coatings or buffer layers, and micro-structure engineering solutions at the electrode/electrolyte interfaces.  \cite{kim2021solid}

There are several critical issues related to the pairing of solid electrolytes with cathode and anode materials, which need to be addressed for long-term battery operation. Solid-state batteries are currently not capable of reliable cycling at current densities $>$ 0.6 mA cm$^{-2}$\cite{famprikis_fundamentals_2019, Albertus2018}. The current density and stability is limited by: poor electrode/electrolyte physical contact leading to particle cracking and interface delamination, formation and propagation of Li dendrites, chemical and electrochemical stability, and high interfacial resistance: \cite{famprikis_fundamentals_2019} 

\begin{itemize}
    %lattice miss-match
    \item The limited time-frames of atomistic modelling are not sufficient to capture lattice relaxation, which allows a coherent (completely matched) interface to form. This amplifies the effects of lattice strain in the model, particularly in cases where periodic boundary conditions are used. \cite{Lepley2015} The lattice strain energy can be calculated and factored into bulk scale calculations but it is not as accurate as explicitly calculating dislocation defects that naturally relieve lattice strain.\cite{Rodney2017, Clouet2020}
    %Dendrite formation
    \item Dendrite formation has been a notable problem for even the most physically robust electrolytes (c.f. section~\ref{sec:se_oxides}). Modelling of dendrite formation mechanisms has yielded some contradictory results due to incomplete models of the interface \cite{Tian2018, Gao2020, Canepa2018}. However, a more detailed understanding requires modelling of larger systems, encompassing the interface and bulk regions of both materials. The entails a high computational cost not reachable through first principles methods. Further development of the linear-scaling DFT approach (c.f. section~\ref{sec:lsdft}) may allow more complete multi-scale approach.
    %electric double layer
    \item The system size limitations in DFT modelling also hinder the modelling of the full electric double layer, which is also applicable to liquid electrolytes. Comparatively, in solid electrolytes the double layer is less understood. For example, \citeauthor{Tateyama2019} were only able to successfully model the initial capacitance layer at the interface (Helmholtz layer).\cite{Tateyama2019}
    %interfacial resistance
    \item Interfacial resistance presents an interesting challenge as it can be introduced through multiple mechanisms\cite{Jiang2019}: electric double layer \cite{Tateyama2019}, surface crystal orientation\cite{Okuno2020}, and production issues such as poor wettability\cite{Sharafi2017}. Strong collaboration between theorists and experimentalists will be needed in order to make informed improvements to current interfacial structures.
\end{itemize}

The interface is the primary source of dendrite formation, lattice mismatch, and interfacial resistance in solid electrolytes. The interface also presents opportunities for atomistic modelling with the growing popularity of coatings that try to address the shortcomings of popular solid electrolytes.\cite{Kim2020, Xu2018exp, Chen2020se_coat, Ito2017, Yin2020, Ji2020coating, Li2020coating, Yi2021coating, Dai2021coating, Pan2020coating, Jing2020coating, Wang2021coating, Zhao2020coating, Zhao2021coating, Liang2020coating, Zhang2020coating} For example, \citeauthor{Tian2018}'s solution to dendrite growth in LLZO by utilising a LiPON coating\cite{Tian2018} (c.f. section \ref{sec:se_oxides}). Understanding how effective coatings are at addressing the aforementioned issues is essential. \cite{Zhang2020directvis, Xiao2019coat, Tian2018} A very recent review by \citeauthor{kim2021solid} presents a detailed insight into the challenges and future prospects of solid-state Li-metal batteries, which we have touched upon here.\cite{kim2021solid}
\end{document}