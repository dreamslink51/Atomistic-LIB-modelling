\documentclass[../main.tex]{subfiles}

\begin{document}

\section{Outlook (Jacqueline)}
In this review, we have introduced the key methods deployed in battery modelling at the atomistic scale and summarised progress in this field, covering models for anodes, cathodes, and both liquid and solid electrolytes. There are several notable developments in atomistic methods for batteries which need to be addressed. These include development of a semi-grand canonical framework incorporating order parameters, with initial promising work under developed by \citeauthor{VanderVen2020},\cite{VanderVen2020} the expansion of the linear scaling DFT codes to link up with kinetic Monte Carlo, the inclusion of entropy effects by parameterising a phase field model from Monte Carlo calculations, development of more accurate force field potentials, and parallelisation to speed up Monte Carlo calculations. 

Alongside deepening our understanding of atomic structure and processes, atomistic models can be used to aid the design of new materials with improved capacity, rate capability, and stability. Multi-scale modelling approaches have been shown to be strong tools to develop novel nanostructures and composites, understand dynamics and phase behaviour, and could eventually allow development of novel interfaces to accommodate volume expansion in silicides. Promising areas for future work include tuning the morphology and composition of the graphite edge  \cite{peng2020lithium,weydanz1994behavior,way1994effect} and interlayer spacing \cite{JI201866} to aid intercalation, optimisation of silicides as anode materials, and investigation of the emerging class of Li-rich cathode materials.

We have identified several outstanding challenges for further work. The hysteresis evident in charge-discharge cycles for both anodes and cathodes, as well as the difference between expected equilibrium potentials and experimentally observable behaviour reveals some grey area in the understanding of path dependency in phase changes and the role of metastable states.

More work is needed to incorporate heterogeneities which form during material synthesis into atomistic models, such as point defects and grain boundaries, to determine what effect these have on the battery performance. Modelling of the complex behaviour at interfaces, in particular, the Solid-Electrolyte Interphase (SEI) for Li-ion batteries with liquid electrolytes and lattice mismatch in solid state batteries is also a challenging area which needs further investigation. Atomistic models have already provided insight into particular aspects of degradation, leading to design of more robust materials, but the development of a universal framework for simulating degradation mechanisms and their interactions would be of great benefit. In order for such a framework to be truly multi-scale, significant work is needed to connect the modelling scales, linking atomistic to continuum modelling and beyond, to control models, as well as forming closer links with experiments at all scales.

\end{document}

