\documentclass[../main.tex]{subfiles}

\begin{document}

\section{Introduction (Mike/Lucy/Julian)}
\label{sec:intro}

% Mike to add a slightly more detailed breakdown ahead of 19th January 2021.
% Key audiences to reach:
% We will try to reach a somewhat wider group of people here in the introduction
% Note: these parts above are on the intended audience, based on an earlier discussion between Mike, Lucy, Chris and Jacqueline

Mathematicians \& continuum modellers new to batteries - yes, we want to introduce them to the concepts applicable to batteries. Reach in intro and conclusions, but not the main body.

DFT users: already plenty of reviews out there. But there are other atomistic modellers we can reach:
a) DFT developers
b) Applying DFT to modellers on other length scales – we have already made this link in several places
c) Atomistic models other than DFT: Monte carlo, MD, etc.

All of these have not been covered so much in other reviews.
Additionally, not all in one place. We reference a lot of other specific reviews. I’m wondering if adding an appendix with a list highlighting other recent reviews we think are prevalent in specific areas. Sort of a go-to guide. i.e.  “for more on DFT of cathode materials, see reviews A, B, and C” type thing.

> 4, Battery modellers at other scales - yes, these are our primary target. 

Clarified above that “atomistic” can be broken down into different sub-length scales.

- two different continuum scales: reach in outlook and highlight the link between atomistic-continuum.
- 1. obtainable for linear scaling DFT, Poisson-Boltzmann approach
- 2. classical (Newman) approach
Like atomistic (DFT, classical), continuum also has different scales. In liquid electrolytes, Arihant’s section reaches the lower lengths, and maxim’s reaches the longer lengths.
Link to control modellers (atomistic $\to$ control) is made in the anodes section already. Make point in the anodes breakdown in introduction.
These links should be made stronger in the text, but also discussed as a point in the overall outlook.

> 5. Other battery experts - yes, these are our target. Like 3, could be broken down as (not exhaustively):

- Battery experimentalists in materials development – most of the other Faraday projects
- Theorists in materials discovery
- Electrochemical engineers – Greg, Dave Howey etc.

> 6. Scientists new to batteries (the "battery curious") - yes, we want to give them a rough impression of the field and its importance. Reach in intro and conclusions but don’t try to cater in main text
My comment for 3 could also cater to these, without adding rephrasing the main text, and without too much additional work.

\textbf{Who to highlight where}

\textbf{Introduction}: focuses on introducing newcomers to batteries and/or atomistic modellers (e.g. new PhD students, postdocs switching fields, etc.), preach to DFT users (a-c), and give one/two sentences link with experiment and continuum models. We should provide a breakdown of the review structure, highlighting there who we think might be interested in each of the main sections.

\textbf{Main text}: Audience will necessarily differ in each subsection but we can clearly signpost in the introduction which part is relevant to whom so that the right people can find the relevant bits. Important, as the majority of readers will not read the whole review.

\textbf{Outlook}: mirrors the introduction in terms of audience and will be broken down into 5-6 bullets as outstanding challenges. One of these will be the link to the different continuum and control models. Link to experiment might need to be a separate point.


*******************************************************************************



Lithium-ion (Li-ion) batteries were first commercialised by Sony in 1991 and have therefore been on the market for nearly 30 years. They are ubiquitous in portable electronic devices, are emerging in hybrid and all-electric vehicles, and could potentially play a role in stationary storage. Despite having been available for such a long duration, not all factors dictating their capacity, performance, safety, and longevity are completely understood. The complexity of battery systems makes it time consuming and impractical to directly measure all of their physical attributes. The grand challenge is to construct a multiscale model, incorporating inputs across length and time scales, that can not only describe, but also predict, changes in behaviour.

To build a truly predictive modelling framework, a physical underpinning to battery models is required, incorporating physically correct descriptions of thermodynamic and kinetic battery behaviour. With sufficient accuracy built in, these models can provide insights on difficult-to-measure internal states, such as electrolyte and ionic concentrations. By contrast, empirical models, which fit a curve to experimental data, are widely used in battery research, but have only a limited physical basis or, in some cases, no physical basis at all. An example is the family of models known as equivalent circuit models, which are widely used in industry, but cannot be relied upon to predict battery behaviour over several charge-discharge cycles.

Physics-based continuum models attempt to describe the behaviour of the whole cell, an example being the Doyle-Fuller-Newman (DFN) model \cite{Newman1975}, dominant in battery modelling for decades. These models need to use simplifications to enable them to run in real time, but their accuracy can be greatly improved by adopting parameters measured using the more detailed, microscopic simulations. Atomistic models are key to building truly physics-based models and form the foundation of the multiscale modelling chain, leading to more robust, higher throughput and predictive models.
 
Atomistic models are not just research tools and determining parameters is not their only value. These models can also be used to predict new behaviour not currently accessible by experiment, for reasons of cost, safety or throughput. They can be used to optimise experimental design and use resources more efficiently, determining whether particular experiments are even worth performing. Atomistic models are useful for quantifying and evaluating trends in experimental data, explaining structure-property relationships, and informing materials design strategies and libraries.

The family of atomistic models itself represents a range of different length and time scales, from the level of electronic structure calculations through conventional and linear scaling density functional theory (DFT), to longer length scales like \textit{ab initio} and classical molecular dynamics (MD), grand canonical Monte Carlo (MC) and kinetic Monte Carlo (kMC). These techniques, along with some recent method developments and some battery-specific observable properties, are summarised in the methods section, section~\ref{sec:methods}. Specific applications to anodes, liquid and solid electrolytes, and cathodes are broken down in the following sections. Links between different methodologies are emphasised, and this review may thus be of particular interest to those looking, for example, to link DFT calculations to Monte Carlo calculations, or apply linear scaling DFT to MD, bridging possible gaps in nomenclature at different length scales. (ACTION: CHRIS, ARIHANT, or anyone in method development - recent related reviews to be mentioned here)

TODO: DIAGRAM AND BREAKDOWN OF KEY COMPONENTS OF LI-ION BATTERY. ALSO TO INCLUDE SOLID STATE ELECTROLYTES. The rest of the review is structured like the components of the battery: anodes, electrolytes, and cathodes.

The anodes section, section~\ref{sec:anodes}, heavily focuses on graphite, the predominant anode material in Li-ion cells. The section describes atomistic modelling of bulk graphite, graphite edges where initial Li-ion insertion occurs, and the solid-electrolyte interphase (SEI). The bulk modelling discussion includes a direct comparison between experimental and theoretical thermodynamic parameters, such as  open circuit voltage and entropy, which will also be of interest to battery control modellers. Kinetic predictions are made and linked to predictions of the influence of graphite edge morphology on surface states using DFT, which may be of interest to those working on battery material development and discovery. Recent work applying linear scaling DFT to challenging interfaces like the SEI will be of interest to those at the forefront of DFT method development, focusing on the boundary between atomistic and continuum modelling. Lastly, recent developments in silicides to boost anode gravimetric capacity, along with their associated challenges, are summarised. Recent reviews in this area include \citeauthor{asenbauer_success_2020} \cite{asenbauer_success_2020}, summarising aspects of lithiation/delithiation mechanisms and morphological aspects in graphite and silicon oxide composites, and \citeauthor{ZHANG2021147} \cite{ZHANG2021147}, similar in scope but providing a more \textit{ab initio} focus. Our review here covers both aspects, including surfaces and interfaces, which have tended to be neglected, although aspects of modelling the SEI have been reviewed by \citeauthor{Wang2018} \cite{Wang2018}.

% ELECTROLYTES SUMMARY: The electrolytes section (delegate Arihant/Chris, maybe Lucy to have first go at solid electrolytes)
Solid electrolytes (SE) are becoming increasingly popular, with the rise of the electric vehicle. In contrast to anodes, there is a wide range of candidate SE materials. In section~\ref{sec:solid_electrolytes}, we review some of the more promising candidates currently being researched. Each material discussed has a different focus, highlighting a range of research areas which may also be applicable to other SE materials. Here, we have focused on four material families, grouping them into sulfide and oxide based SEs. Sulfide based SEs typically have a high Li-ion conduction and poor electrochemical stability against Li metal (the anode typically used for SEs). \cite{Zhu2015, Zhang2019se_rev} Li$_{10}$GeP$_2$S$_{12}$ (LGPS) is reviewed with a focus on how atomistic methods revealed the isotropic ion pathways, while Li$_6$PS$_5$\textit{X} based Li-argyodites are focused towards the relationship between ionic conductivity and anion substitution, as well as atomistic predictions of occupied Li sites. The oxide section reviews Li$_7$La$_3$Zr$_2$O$_2$ (LLZO) and oxide nanocomposites. Oxides typically have a higher electrochemical stability but still suffer from dendrite formation, among other issues.\cite{Zhu2015} LLZO is used to demonstrate how multiple atomistic methods have been applied to probe dendrite formation and ionic transport in the material, while oxide nanocomposites are used to probe the interfaces between the nanosized particles. Finally, we discuss SE interfaces, which are a huge challenge for all-solid-state batteries and therefore have been the subject of many recent atomistic investigations. Recent related reviews on solid electrolytes include \citeauthor{Sun2017se_review} \cite{Sun2017se_review}, providing a general review of solid-state batteries, and \citeauthor{Zhang2018se_review}, \cite{Zhang2018se_review} discussing the future directions of solid state batteries. \citeauthor{famprikis_fundamentals_2019} \cite{famprikis_fundamentals_2019} has highlighted the advances in the fundamental understanding of SEs and solid-state batteries, with \citeauthor{Gurung2019} \cite{Gurung2019} providing details on the challenges faced when using SEs in a battery. More specific reviews of the SE interfaces include \citeauthor{Xu2018exp} \cite{Xu2018exp} and \citeauthor{Xiao2020interfacerev} \cite{Xiao2020interfacerev}, exploring the challenges and stability, and \citeauthor{Tateyama2019} \cite{Tateyama2019}, summarising the cathode-SE interface from a theoretical standpoint. From a modelling perspective, \citeauthor{Ceder2018} \cite{Ceder2018} outlines the principles that should be employed when modelling SEs and \citeauthor{Gao2020_ion_transport} \cite{Gao2020_ion_transport} discusses the techniques used for investigating ionic transport.
% \citeauthor{Samson2019}'s \cite{Samson2019} detailed look at LLZO with a focus on experiment and \citeauthor{Zhang2019se_rev}'s investigate the current state of sulfide SEs,

% CATHODES SUMMARY: The cathodes section (delegate Lucy for first draft, other cathodes people to contribute too). Plenty of reviews focused on cathodes - cite a few key ones and mention how ours differs
The cathodes section, section~\ref{sec:cathodes}, covers a range of different cathode materials used in a variety of Li-ion cells. The section describes atomistic modelling in the bulk, on the surfaces of some cathode materials, and the cathode-electrolyte interphase (CEI). In discussing bulk modelling, a comparison of the different cathode crystal structures, micro-structuring, and available diffusion pathways within the material are covered, as well as important properties, including redox and electronic properties, transition metal ordering, and vibrational and thermal properties. Cathode crystal structure is a key influence on ion diffusion. Use of first-principles modelling techniques has been essential for investigating these properties, so will be of great interest to those who utilise DFT in their research. Surface structures and morphologies of cathode particles can be difficult to determine using experimental methods alone, which is where \textit{ab initio} and classical molecular dynamics can provide vital insight. As with the SEI, linear scaling DFT has recently been applied to CEI, where discussions on CEI will be of interest to those doing state-of-the-art DFT method development. Related reviews in the area include \citeauthor{ma2018computer}'s \cite{ma2018computer} general review on modelling Li-ion battery cathode materials, \citeauthor{yan2014review}'s \cite{yan2014review} more specific review which focuses only on first-principles calculations of cathode materials, and \citeauthor{radin2017narrowing}'s \cite{radin2017narrowing} and \citeauthor{wang2018reviving}'s \cite{wang2018reviving} reviews which discuss closing the gap between theoretical and practical capacities in layered-oxide cathode materials. There is also a slightly older review by \citeauthor{daniel2014cathode} \cite{daniel2014cathode} which gives an overview of cathode materials. Our review includes a discussion on the CEI, which has recently been reviewed by \citeauthor{maleki2019controllable} \cite{maleki2019controllable}.

Finally, we provide an outlook on the key remaining challenges for atomistic modelling of Li batteries and promising future directions for resolving them.


% \begin{itemize}
%    \item Ways in which atomistic results can be fed into higher scales. Where atomistic models fail and need input from other length scales. (Touch on parameterisation from experiment as well, maybe a sentence referencing the parameterisation review?)
%    \item \textbf{} Description of Li-ion battery, (+schematic high level diagram) - possibly general alkali-ion battery. Solid state electrolyte battery as separate figure.
%    \item Review of literature, especially other reviews. There is a recent review by Anton Van der Ven and other big names, and another in 2019 so it would be good to check we aren't overlapping with these too. Some earlier reviews to read: \citet[2001]{VanderVen2001}, \citet[2017]{radin2017narrowing}, and \citet[2020]{VanderVen2020}
%    \item objective, relevance, novelty or importance of the current review.
%    \item Description of the outline
%        \begin{itemize}
%           \item Description of methods: first draft done, 
%           \item Anodes: for a recent overview add
%           \item Electrolytes
%           \item Cathodes
%           \item Outlook
%        \end{itemize}
%\end{itemize}






% Batteries can be simply defined as an electrochemical power source that converts chemical energy to electrical energy. Originally, this electrochemical reaction was irreversible, and therefore the battery could only discharge once, and not be recharged. These were known as ``primary'' batteries and were commercialised around 30 years ago. Some of the best examples of these consisted of zinc and copper electrodes (Daniell element), \cite{daniel2014cathode} carbon-zinc batteries, \cite{linden2002handbook} and zinc-air batteries. \cite{li2013advanced} ``Secondary'' batteries are where the electrochemical reaction is reversible, and therefore rechargeable for repeated use. Some examples of secondary batteries are lead-acid batteries, nickel-cadmium batteries, and lithium-ion batteries (LIBs). \cite{linden2002handbook} Lead-acid batteries were invented by Plant\'{e} in 1859 and are widely determined to be the first rechargeable batteries, with a specific power of 180 W/kg and an efficiency of 60-90 \%. Even though the material and production cost was low, their working temperature was limited, with the electrolyte freezing at winter temperatures. Since this time, a wide range of battery materials have been explored, increasing the specific capacity, working temperature range, and improving weight and toxicity limitations and concerns. In 1976, Whittingham proposed the first battery that was termed lithium-ion, consisting of titanium(IV) sulfide and lithium metal, \cite{whittingham1976electrical} with a graphite anode and oxide cathode LIB proposed later in the same year by Besenhard. \cite{besenhard1976electrochemical} LIBs became more promising in 1979 when Goodenough and Mizushima successfully showed LiCoO$_2$ as a cathode, highlighting that an anode material other than lithium metal could be used. \cite{mizushima1980lixcoo2} LIBs can now reach specific powers of 250-340 W/kg, with an efficiency of 90 \%, thus making LIBs the most efficient rechargeable battery in portable electronics. \cite{whittingham2008materials}

\end{document}