\documentclass[../main.tex]{subfiles}

\begin{document}

\section{Outlook (Jacqueline)}
In this review, we have ...


\begin{itemize}

    \item Further developments in atomistic methods of batteries: some examples mentioned in the anodes outlook - linking Onetep with kMC, etc, use of GPUs to parallelise and speed up Monte Carlo calculations, superior force field potentials (mentioned in cathodes).
    \item Materials design for improved capacity, rate capability, stability. Mentioned in all 3 sections. For example, edge doping and morphology tuning of graphite, silicides, excess Li cathodes
    \item Path dependency and connectivity of phase space: hysteresis (difference between charge and discharge behaviour) is a universal challenge for anodes and cathodes, and the role of metastable behaviour is an outstanding challenge. This results in challenges in comparing experimentally observable behaviour like the OCV with equilibrium potentials, for example. 
    \item Heterogeneities: point defects can be done, grain boundaries challenging - mentioned in cathodes and solid electrolytes.
    \item \textbf{Interfaces}: mentioned in anodes (SEI), lattice mismatch in solid state batteries, briefly mentioned in cathodes (CEI). could be mentioned last as possibly to biggest frontier
    \item A universal framework to understand degradation is still missing, but atomistic models can help to understand particular aspects and develop more robust materials
    \item Linking atomistic modelling to continuum and control models and/or experiment. - experimental and continuum link mentioned in anodes, control link mentioned in anodes. (experiment might need to be separate)

\end{itemize}

\end{document}