\documentclass[../main.tex]{subfiles}

\begin{document}

\section{Introduction (Mike/Lucy/Julian)}
\label{sec:intro}

% % Key audiences to reach:
% % We will try to reach a somewhat wider group of people here in the introduction

% Mathematicians \& continuum modellers new to batteries - yes, we want to introduce them to the concepts applicable to batteries. Reach in intro and conclusions, but not the main body.

% DFT users: already plenty of reviews out there. But there are other atomistic modellers we can reach:
% a) DFT developers
% b) Applying DFT to modellers on other length scales – we have already made this link in several places
% c) Atomistic models other than DFT: Monte carlo, MD, etc.

% All of these have not been covered so much in other reviews.
% Additionally, not all in one place. We reference a lot of other specific reviews. I’m wondering if adding an appendix with a list highlighting other recent reviews we think are prevalent in specific areas. Sort of a go-to guide. i.e.  “for more on DFT of cathode materials, see reviews A, B, and C” type thing.

% > 4, Battery modellers at other scales - yes, these are our primary target. 

% Clarified above that “atomistic” can be broken down into different sub-length scales.

% - two different continuum scales: reach in outlook and highlight the link between atomistic-continuum.
% - 1. obtainable for linear scaling DFT, Poisson-Boltzmann approach
% - 2. classical (Newman) approach
% Like atomistic (DFT, classical), continuum also has different scales. In liquid electrolytes, Arihant’s section reaches the lower lengths, and maxim’s reaches the longer lengths.
% Link to control modellers (atomistic $\to$ control) is made in the anodes section already. Make point in the anodes breakdown in introduction.
% These links should be made stronger in the text, but also discussed as a point in the overall outlook.

% > 5. Other battery experts - yes, these are our target. Like 3, could be broken down as (not exhaustively):

% - Battery experimentalists in materials development – most of the other Faraday projects
% - Theorists in materials discovery
% - Electrochemical engineers – Greg, Dave Howey etc.

% > 6. Scientists new to batteries (the ``battery curious'') - yes, we want to give them a rough impression of the field and its importance. Reach in intro and conclusions but don’t try to cater in main text
% My comment for 3 could also cater to these, without adding rephrasing the main text, and without too much additional work.

% \textbf{Who to highlight where}

% \textbf{Introduction}: focuses on introducing newcomers to batteries and/or atomistic modellers (e.g. new PhD students, postdocs switching fields, etc.), preach to DFT users (a-c), and give one/two sentences link with experiment and continuum models. We should provide a breakdown of the review structure, highlighting there who we think might be interested in each of the main sections.

% \textbf{Main text}: Audience will necessarily differ in each subsection but we can clearly signpost in the introduction which part is relevant to whom so that the right people can find the relevant bits. Important, as the majority of readers will not read the whole review.

% \textbf{Outlook}: mirrors the introduction in terms of audience and will be broken down into 5-6 bullets as outstanding challenges. One of these will be the link to the different continuum and control models. Link to experiment might need to be a separate point.


% *******************************************************************************



Lithium-ion (Li-ion) batteries (LIBs) were first commercialised by Sony in 1991. \cite{zeng2019commercialization} They are ubiquitous in portable electronic devices, are emerging in hybrid and all-electric vehicles, \cite{Goodenough2010} and could potentially play a role in large-scale stationary storage. \cite{kubiak2017calendar} Despite over 30 years of development and commercialisation, not all factors dictating their capacity, performance, safety, and longevity are completely understood. The complexity of battery systems makes it time consuming and impractical to directly measure all of their physical attributes. The grand challenge is to construct a multiscale model, incorporating inputs across length- and time-scales that can not only describe, but also predict, changes in behaviour.

To build a truly predictive modelling framework, a physical underpinning to battery models is required, incorporating physically correct descriptions of thermodynamic and kinetic battery behaviour. With sufficient accuracy built in, these models can provide insights on difficult-to-measure internal states, such as electrolyte and ionic concentrations. By contrast, empirical models, which fit a curve to experimental data, are widely used in battery research, but have only a limited physical basis or, in some cases, no physical basis at all. For example, equivalent circuit models, which are widely used in industry, cannot be relied upon to predict battery behaviour over several charge-discharge cycles.

Physics-based continuum models attempt to describe the behaviour of whole cells, for example the widely used Doyle-Fuller-Newman (DFN) model. \cite{doyle1993modeling,fuller1994simulation, Fuller1994a,Doyle1995,Newman2004} These models need to use simplifications to enable them to run in real time, but their accuracy can be greatly improved by adopting parameters measured using more detailed, microscopic simulations. Atomistic models are key to building truly physics-based models and form the foundation of the multiscale modelling chain, leading to more robust and predictive models.
 
Atomistic models are research tools with high predictive accuracy. For example, they can be used to predict new behaviour not currently accessible by experiment, for reasons of cost, safety, or throughput. They can be used to optimise experimental design and use resources more efficiently, determining whether particular experiments are even worth performing and also provide unique insights into the behaviour of materials that may not even be accessible, or are impractical to obtain, by experimental probes. Atomistic models are useful for quantifying and evaluating trends in experimental data, explaining structure-property relationships and informing materials design strategies and libraries.

% METHODS SUMMARY
The family of atomistic models itself represents a range of different length- and time-scales, from the level of electronic structure calculations through conventional and linear scaling Density Functional Fheory (DFT), to \textit{ab initio} Molecular dynamics (MD) and on to longer length scale models, such as potentials-based MD, Monte Carlo (MC), and kinetic Monte Carlo (kMC) calculations, which can be parameterised either from classical force field potentials or using \textit{ab initio} calculations. These techniques, along with recent method developments and battery-specific observable properties, are summarised in the methods section of this review, section~\ref{sec:methods}. Specific applications to anodes, liquid and solid electrolytes, and cathodes are broken down in the following sections. Links between different methodologies are emphasised and this review may thus be of particular interest to those looking, for example, to link DFT calculations to MC calculations, or apply linear scaling DFT to MD, bridging possible gaps in nomenclature at different length scales. Atomistic models linking to \textit{ab initio} calculations are summarised by \citeauthor{VanderVen2020} \cite{VanderVen2020}; also noteworthy in this area is a review by \citeauthor{Shi_2016},\cite{Shi_2016} and an older review by \citeauthor{franco2013multiscale}.\cite{franco2013multiscale} A review of method development in the area of hybrid quantum-continuum solvation models is presented by \citeauthor{Tomasi2005}.\cite{Tomasi2005}

%TODO: DIAGRAM AND BREAKDOWN OF KEY COMPONENTS OF LI-ION BATTERY. ALSO TO INCLUDE SOLID STATE ELECTROLYTES. The rest of the review is structured like the components of the battery: anodes, electrolytes, and cathodes.

% ANODES SUMMARY
The anodes section, section~\ref{sec:anodes}, heavily focuses on graphite, which is still the predominant anode material in Li-ion cells. The section describes atomistic modelling of bulk graphite, graphite edges where initial Li-ion insertion occurs, and the Solid-Electrolyte Interphase (SEI). The bulk modelling discussion includes a direct comparison between experimental and theoretical thermodynamic parameters, such as the open circuit voltage (OCV) and entropy, which will also be of interest to battery control modellers. Kinetic predictions are made and linked to DFT predictions of the influence of graphite edge morphology on surface states, which may be of interest to those working on battery material development and discovery. Recent work applying linear scaling DFT to complex interfaces, such as the SEI, will be of interest to those at the forefront of DFT method development, focusing on the boundary between atomistic and continuum modelling. Lastly, recent developments in silicides to boost anode gravimetric capacity, along with their associated challenges, are summarised in the outlook. Recent reviews in this area include \citeauthor{asenbauer_success_2020} \cite{asenbauer_success_2020}, summarising aspects of lithiation/delithiation mechanisms and morphological aspects in graphite and silicon oxide composites, and \citeauthor{ZHANG2021147} \cite{ZHANG2021147}, similar in scope but providing a more \textit{ab initio} focus. Here, our review here covers graphite structure and lithiation/delithiation mechanisms, including surfaces and interfaces, which have tended to be neglected, although aspects of modelling the SEI have been reviewed by \citeauthor{Wang2018} \cite{Wang2018}.

% LIQUID ELECTROLYTES SUMMARY
The liquid electrolyte section, section~\ref{sec:Liquid_electrolytes}, has a strong focus towards the development of atomistic models. This includes a pivotal discussion on the atomic interactions between components and the development in the ways they can be represented in MD simulations. This will be of particular interest to those at the forefront of classical MD method development. The latter parts of this section describe the atomistic modelling of the bulk structure and landscaping, Li-ion diffusion, solvation energies, and activity coefficients of liquid electrolytes, and the interfacial nanostructure relating to the interface with a solid electrode. These topics cover the major aspects for improving liquid electrolytes for use in a battery and research towards circumventing critical safety\cite{Shepherd_Siddiqui, Pfrang2017} and energy density\cite{Liu2019_e_den} limitations. Liquid electrolytes are know to be limited by narrow electrochemical windows, solvent toxicity, and material flammability/safety concerns. The challenges and potential avenues for solving these issues are summarised in the outlook, including recent developments to solving these within the liquid electrolyte family and alternative materials. Recent reviews in this area include........ Here, our review covers the continued development of interatomic potentials for liquid electrolytes and a description of the solid electrode-liquid electrolyte interface from the perspective of the liquid, which is not the conventional frame of reference.

% SOLID ELECTROLYTES SUMMARY
Solid electrolytes (SEs) are becoming an increasingly popular avenue of research, alongside the rise of the electric vehicle (EV), \cite{Woods_2021} as an alternative to liquid electrolytes as a solution to the safety issues pertaining to the flammable electrolyte\cite{Shepherd_Siddiqui, Pfrang2017} and as a route to increased energy density\cite{Liu2019_e_den}. In the solid electrolyte section, section~\ref{sec:solid_electrolytes}, we review a selection of the promising candidate materials currently being researched. Each material discussed has a different focus, highlighting a range of properties applicable to a range of SE materials. In this section, we focus on four material families, grouping them into sulfide and oxide based SEs. Sulfide based SEs typically have a high Li-ion conduction and poor electrochemical stability against Li metal (the anode typically used for SEs). \cite{Zhu2015, Zhang2019se_rev} Li$_{10}$GeP$_2$S$_{12}$ (LGPS) is reviewed with a focus on how atomistic methods revealed the isotropic ion pathways, while Li$_6$PS$_5$\textit{X} based Li-argyrodites are focused towards the relationship between ionic conductivity and anion substitution, as well as atomistic predictions of occupied Li sites. Oxides typically have a higher electrochemical stability but still suffer from dendrite formation, among other issues.\cite{Zhu2015} Li$_7$La$_3$Zr$_2$O$_2$ (LLZO) is reviewed with a focus on how multiple atomistic methods have been applied to probe dendrite formation and ionic transport in the material. Oxide nanocomposite are used to highlight modelling of interfaces between nanosized particles. Lastly, the challenges of SEI are discussed and an outlook to future modelling of SEs is given. Related reviews in the area include \citeauthor{Zhang2018se_review}, \cite{Zhang2018se_review} summarising the future directions of solid state batteries, and \citeauthor{Gurung2019}, \cite{Gurung2019} highlighting the advances and challenges in SEs and solid state batteries. \citeauthor{Xiao2020interfacerev}\cite{Xiao2020interfacerev} and others\cite{Xu2018exp,Tateyama2019} provide a more specific review of the SEI. \citeauthor{Ceder2018} \cite{Ceder2018} outlines the principles that should be employed when modelling SEs. Here, our review discusses a broad range of SE properties with the idea these can be applied to a range of materials, where they are usually discussed in terms of a specific material.

% CATHODES SUMMARY
The cathodes section, section~\ref{sec:cathodes}, covers a range of different cathode materials used in a variety of Li-ion cells. This section describes atomistic modelling in the bulk, at the surfaces, and the Cathode-Electrolyte Interphase (CEI). In discussing bulk modelling, a comparison of the different cathode crystal structures, micro-structuring, and available diffusion pathways within the material are covered, as well as important properties, including redox and electronic properties, transition metal ordering, and vibrational and thermal properties. Use of \textit{First Principles} modelling techniques has been essential for investigating crystal structure, so will be of great interest to those who utilise DFT in their research. Surface structures and morphologies of cathode particles can be difficult to determine using experimental methods alone, which is where \textit{ab initio} and potentials-based MD can provide vital insight. As with the SEI, linear scaling DFT has recently been applied to CEI, where discussions on CEI will be of interest to those doing state-of-the-art DFT method development. Related reviews in the area include \citeauthor{ma2018computer}, \cite{ma2018computer} summarising modelling Li-ion battery cathode materials, \citeauthor{yan2014review}, \cite{yan2014review} focusing on \textit{First Principles} calculations of cathode materials, and \citeauthor{wang2018reviving}, \cite{wang2018reviving} discussing closing the gap between theoretical and practical capacities in layered oxide cathode materials. Our review includes a discussion on the CEI, which has recently been reviewed by \citeauthor{maleki2019controllable}. \cite{maleki2019controllable} Here, our review covers thermal, electronic, dynamic, and structural properties for a range of prominent cathode materials in terms of both \textit{first principles} and potential-based modelling, which have tended to be more isolated in other reviews.

Finally, we provide an outlook on the key remaining challenges for atomistic modelling of LIBs and promising future directions for resolving them.

\end{document}