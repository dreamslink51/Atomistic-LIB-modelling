\documentclass[../main.tex]{subfiles}

\begin{document}

\section{Outlook}
In this review, we have introduced the key methods deployed in battery modelling at the atomistic scale (section~\ref{sec:methods}) in lithium-ion batteries (LiBs) and solid-state batteries (SSBs), which are collectively called lithium batteries. We have summarised progress in this field, covering models for anodes, liquid and solid electrolytes, and cathodes. Outlooks specific to these components are summarised as follows: anodes, in section~\ref{sec:anodes_outlook}; liquid electrolytes, in section~\ref{sec:liquids_outlook}; solid electrolytes, in section~\ref{sec:solids_outlook}; and cathodes, in section~\ref{sec:cathodes_outlook}.

There are several notable developments in atomistic methods for lithium batteries which need to be addressed. These include development of a semi-grand canonical framework, incorporating order parameters, with initial promising work developed by \citeauthor{VanderVen2020},\cite{VanderVen2020,vanderven2018} the expansion of the linear scaling Density Functional Theory (DFT) codes,\cite{Goedecker1999,ONETEP2005,ONETEP2020} to link up with kinetic Monte Carlo (kMC), the inclusion of entropy effects by parameterising a phase field model (such as those developed by \citeauthor{Bazant2017})\cite{Bazant2017,peng2011,guo2016} using results obtained from Monte Carlo (MC) calculations, development of more accurate force field potentials, and parallelisation to speed up MC calculations on longer length scales.

Alongside deepening our understanding of atomic structure and processes, atomistic models can be used to aid the design of new materials with improved capacity, rate capability, and stability. Multiscale modelling approaches have been shown to be strong tools to develop novel nanostructures and composites, understand dynamics and phase behaviour, and could identify novel interfaces to accommodate volume expansion in solid solution materials, such as silicides. Promising areas for future work include tuning the morphology and composition of graphite edges\cite{peng2020lithium,weydanz1994behavior,way1994effect} and interlayer spacings\cite{JI201866} to aid intercalation, improved understanding of the phase behaviour and dynamics of silicides as anode materials \cite{Jiang_2020}, and investigation of the emerging class of Li-rich cathode materials.\cite{Hy2016,naylor2019depth,House2020}

We have identified several outstanding challenges for further work. For example, certain anode and cathode materials show pronounced hysteresis between charge and discharge cycles. \cite{Liu2014,Assat2019,Mercer2021,Grismann2017,Zheng1996,Jiang_2020} This results in a difference between expected equilibrium potentials from atomic-scale calculations and the experimentally measurable open circuit voltage (OCV), which creates ambiguity when using the measured OCV in longer length scale models, like control models for battery management systems. Future kinetic models must therefore account for metastable behaviour that can persist over experimental time scales of hours or even days.\cite{Liu2014} The next generation of models should consider the connectivity between different phase transformations, with the framework developed by \citeauthor{VanderVen2020} highlighted above being one promising solution that is potentially transferable to a variety of material types.

Flammable liquid electrolyte materials currently dominate the commercial market. Development of safer, non-flammable, electrolyte materials is key to addressing safety concerns and accidents resulting from attempts to confine increasing energy densities into smaller volumes and into geometries that are challenging to thermally manage. More work is needed to investigate potential avenues for resolving these issues, including alternative liquid electrolytes,\cite{Shepherd_Siddiqui, McCurry_2017, Pfrang2017} such as water-in-salt electrolytes,\cite{suo2015water,chen2020water} and replacing liquid electrolytes with solid or soft matter alternatives.\cite{Woods_2021,kim2021solid} Advancements in electrolyte design are crucial, where critical obstacles could be resolved through new novel electrolyte salts and solvents. Development and open source accessibility of atomistic scale models, combined with improved experimental studies, provide a framework for high throughput screening of electrolyte materials.\cite{merlet_highly_2013, borodin_interfacial_2014, Simoncelli_2018,marin-lafleche_metalwalls_2020}

More work is needed to incorporate heterogeneities formed during material synthesis and battery degradation,\cite{Edge2021,Birkl2017} such as point defects \cite{mercer_influence_2017,schlueter_quantifying_2018,squires_2020,Swift2021,hoang2016defect} and grain boundaries,\cite{dean2021overscreening,Kim2020,symington2021elucidating} into atomistic models and to determine their effect on battery performance. Modelling of the complex behaviour at interfaces, such as the solid-electrolyte interphase (SEI) in LiBs and lattice mismatch in SSBs, is a prominent challenge which requires further investigation. Atomistic models have already provided insight into particular aspects of degradation, leading to design of more robust materials, but the development of a universal framework for simulating degradation mechanisms and their interactions would be of great benefit and is still beyond current capabilities. In order for such a framework to be truly multiscale, significant work is needed to connect the modelling scales, linking atomistic to continuum modelling and on to longer length scales, such as control models, as well as forming closer links with experiments at all scales.

This review has focused almost entirely on lithium batteries, given that they currently comprise the most technologically advanced rechargeable battery systems that are commercialised at scale. However, atomistic modelling applied to LiBs also improves understanding of batteries that could be based on more environmentally-friendly or Earth-abundant materials, such as sodium. Solid state models of intercalation, applied to LiBs, are directly transferable to other intercalation chemistries. The understanding of interfaces in batteries with other chemistries is even less developed than in LiBs. However, the modelling frameworks highlighted in this review, such as the linear-scaling DFT framework, could also be applied to improve understanding of these interfaces.
\end{document}
