\documentclass[../main.tex]{subfiles}

\begin{document}

\section{Methods}
\subsubsection{Extracting Stefan-Maxwell diffusivities from MD, and Onsager-Casimir decay of fluctuations hypothesis (MZ)}
\label{sec:stefan-maxwell}

The Onsager decay of fluctuation hypothesis asserts that the autocorrelation functions of fluctuating quantities, in the linearized approximation, satisfy the same equations that the linearization of continuum transport equations provide. Such theory has been developed in \cite{mwn2006,mwn2009,mwn2015}. In continuum theory, material balance equations for species molar concentrations, $c_i$, are given by:
\begin{equation} \begin{array}{l}
 \frac{\partial c_i}{\partial t} = - \nabla \mathbf{J}_i  , \quad
    \mathbf{J}_i  = c_i \mathbf{v}_i ,
    \label{massbalance}
\end{array} \end{equation}
\noindent where $\mathbf{v}_i$ are velocities of species $i$. Thermodynamic equation of state at a constant pressure (PNT ensemble) implies\cite{GoyalMonroe} a relation among molar concentrations, $c_i,$
\begin{equation}
    \sum_{i=1}^n c_i \bar{V}_i ( \, \mathbf{y} ) \, =1,
\end{equation}
\noindent where $\mathbf{y} = (y_1, y_2, \ldots y_{n-1})$ are molar fractions,  $y_j = \frac{c_j}{c_T}$ , $y_n \equiv 1-\sum_{i=1}^{n-1} y_i,$  $\bar{V}_i$ are partial volumes, $\bar{V}_i = \left. \left(\, \frac{\partial V}{\partial n_i} \right)\, \right\vert_{P,T, n_j,j\neq i}$,  and $c_T = \sum_j c_j$ is the total concentration, $\frac{1}{c_T} = \sum_{i=1}^n y_i \bar{V}_i ( \, \mathbf{y} )\label{cT},$ In the nonequilibrium thermodynamics framework\cite{GoyalMonroe}, there is a relationship between generalized forces and fluxes, ensuring that entropy is non-decreasing. %For systems locally near equilibrium at constant temperature and pressure, 
This implies Stefan-Maxwell relationships:
%cite Neuman and Goyal
\begin{equation} \begin{array}{l}
    \displaystyle c_i \nabla \mu_i = \displaystyle \sum_{j\neq i}  K_{ij} (\mathbf{v}_j - \mathbf{v}_i), \
    K_{ij}= R  T  \frac{c_i c_j }{c_T D_{ij}}.
    \label{StM:K}
\end{array} \end{equation}

%\noindent where $D_{ij}= D_{ji},$ $i \neq j,$ are relative diffusion coefficients of species $i$ and $j$. ~\ref{StM:K}  ensures that entropy production rate $g_s$ is non-negative:

%\begin{equation}
   % T g_s = - \sum_{i=1}^n c_i \nabla \mu_i ( \, \mathbf{v}_i - \mathbf{v}_{\mbox{ref}} ) = \frac{R  T}{2} \displaystyle\sum_{ij=1}  \frac{c_i c_j}{c_T D_{ij}} ( \, \mathbf{v}_i -\mathbf{v}_j ) \,^2 \geq 0 , \label{gs}\end{equation}

%\noindent where $\mathbf{v}_{\mbox{ref}}$  is a reference velocity, which does not affect $g_s$ due to the Gibbs-Duhem relations ~\ref{GibbsDu}.

%The ``self-diffusion'' coefficient, where $i=j$, does not appear in this approach to continuum modeling. To link this approach to other approaches one may consider a single particle as a separate species \cite{2002PhDWheeler}.

%It is important to note that
To find fluxes $\mathbf{J}$ in ~Eq.\ref{massbalance} we need to know species velocities, however we {\bf cannot} find all of them
%just invert Stefan-Maxwell relations ~\ref{StM:K} to find all those velocities,
as adding a reference velocity to all the species velocities does not change  Eq.~\ref{StM:K}.
Indeed, ~\ref{StM:K} can be written as a linear system as:
\begin{equation} \begin{array}{l}
    \displaystyle \sum_{j }  M_{ij}  \mathbf{v}_j  = c_i \nabla \mu_i , \
    M_{ij}=  K_{ij}  i\neq j; M_{ij}= \sum_{k\neq i} K_{ik}, i = j; 
    \label{StM:M}
\end{array} \end{equation}
%\noindent where $K_{ij}$ is defined in ~\ref{StM:K}. The matrix $M = \left\{ M_{ij} \right\}$ is a symmetric matrix with zero margins for all rows or columns,
$
M= M^T, \  \displaystyle  \sum_{i} M_{ij} = \sum_{j } M_{ij} = 0.
$
Such $M$ is \textbf{not} invertible, as it has a nullvector $\mathbf{z}_0= (1,1, \ldots 1)^{\mbox{\scriptsize T}}: $
$
\mathbf{z}_0^{T}   \ . \  M = M \ . \ \mathbf{z}_0 = 0.
$
However, due to the Gibbs-Dulem relations \cite{GoyalMonroe} the right-hand side of ~\ref{StM:M} is orthogonal to the nullvector:
$
 \mathbf{z}_0^{T}   \ . \ \left( c  \nabla \mu \right) =   \sum_{i=1}^n c_i \nabla \mu_i = 0.
$
It follows from linear algebra that ~\ref{StM:M} can be solved for velocities, up to  a reference velocity.%; while such a reference velocity cannot be determined from ~\ref{StM:M}.

Since $\sum_{j } M_{ij} = 0$, we can use relative velocities, e.g. velocities relative to the solvent velocity  $\mathbf{v}_0 :$
$
 \sum_{j }  M_{ij} \left(  \mathbf{v}_j - \mathbf{v}_0 \right) = c_i \nabla \mu_i.
     \label{StM:MM}
$
This motivates introducing the Stefan-Maxwell ``self-diffusion'' coefficient $D_{ii}$ by
\begin{equation} \begin{array}{l}
   M_{ii} = -\frac{c_i^2 R  T }{c_T D_{ii}}, \frac{y_i }{ D_{ii}} = \sum_{k\neq i} \frac{y_k }{ D_{ik}}.
\end{array} \end{equation}

%In the literature, a version of Stefan-Maxwell relations resolved for the fluxes is also used, $y_i \mathbf{v}_i  = - \frac{1}{RT} \sum_{j} L_{ij} \nabla \mu_j .\label{StM:L}$
%Extra care is needed in using such pseudo-inverse form, since gradients of chemical potentials are not all independent and must satisfy the Gibbs-Duhem relations, while species velocities are determined up to a reference velocity.

To relate continuum model with molecular dynamics, we need linearized continuum modeling equations. %, remembering that  only $(n-1)$  molar fractions $\left\{ y_i \right\}$ are linearly independent. Since $\sum_{i=1}^{n} y_i =1, $ we can resolve for $y_n,$  $y_n = - \sum_{i=1}^{n-1} y_i.$
Expanding those equations to the first order about their equilibrium values:
%, denoted by $\infty$
$y_i^\infty$ (constants determined by composition), $\mathbf{v}_i^{\infty} = 0,$ we get
\begin{equation} \begin{array}{l}
    y_i = y_i^\infty + y_i^{(1)}+ \ldots, i=1,2\ldots n-1; \quad y_n = - \displaystyle\sum_{i=1}^{n-1} y_i;
    \\
    \mathbf{v}_i = 0 + \mathbf{v}_i^{(1)} + \ldots .
\end{array} \end{equation}

\noindent Linearization of the Stefan-Maxwell equations yields:

\begin{equation} \begin{array}{l}
    \nabla \mu_i^{(1)} \equiv \displaystyle   \frac{RT}{y_i^{\infty}}
    \sum_{j=1}^{n-1} Q_{ij} \nabla  y_j^{(1)}
     = RT \displaystyle \sum_{j=1}^{n}\frac{y_j^{\infty}}{D_{ij}^{\infty}} ( \, \mathbf{v}_j^{(1)} - \mathbf{v}_i^{(1)}) \,
   % \\
   % \nabla \mu_i^{(1)} = \displaystyle \frac{RT}{y_i^{\infty}}\sum_{j=1}^{n-1} Q_{ij} \nabla  y_j^{(1)}  , i
   % =1,2,\ldots n-1,
    \\
    Q_{ij} = \displaystyle \frac{ y_i^{\infty} }{RT} \frac{\partial \mu_i^{\infty} }{\partial  y_j}, \ \mbox{\small $i,j  =1,2,\ldots n-1$}.
    \label{linStefMax}
\end{array} \end{equation}

Linearization of the material balance equations, equation ~\ref{massbalance},  gives:

\begin{equation}\begin{array}{l}
    \displaystyle \frac{\partial  y_i^{(1)}}{\partial t} + \frac{ y_i^{\infty}}{c_T^{\infty} } \frac{\partial  c_T^{(1)}}{\partial t} = - y_i^{\infty}  \nabla \mathbf{v}_i^{(1)},
    \label{linmass}
\\
     c_T^{(1)}  =    \displaystyle \sum_{j=1}^{n-1} \frac{\partial c_T^{\infty}}{\partial y^j} y_j^{(1)},
     \ \frac{\partial c_T^{\infty}}{\partial y^j} = -\frac{1}{( \, c_T^{\infty}) \,^2}  \sum_{i=1}^{n-1} ( \, ( \,  \bar{V}_i^{\infty} - \bar{V}_n^{\infty} ) \,  \delta_{ij} +
    % \sum_{j=1}^{n-1}
    \frac{ y_i^{\infty}\partial  \bar{V}_i^{\infty} }{\partial  y_j}) \,

\end{array} \end{equation}

Differentiating equation ~\ref{linStefMax} and eliminating $\nabla \mathbf{v}$ from equation ~\ref{linmass} gives:
%Differentiating equation~\ref{linStefMax}  and eliminating $\nabla \mathbf{v}$ from equation~\ref{linmass}  gives:
\begin{equation} \begin{array}{l}
     \sum_{j=1}^{n-1} R_{ij} \frac{\partial  y_j^{(1)}}{\partial t}= - \sum_{j=1}^{n-1} Q_{ij} \nabla^2 y_j^{(1)},
    \label{eq:y1}
\\
     R_{ij} =   \frac{ y_i^{\infty}}{D_{ij}} - \frac{ y_i^{\infty}}{D_{in}}, i\neq j;
   R_{ij}  =
- (   \frac{ y_i^{\infty}}{D_{in}} + \sum_{k\neq i}^{n} \frac{ y_k^{\infty}}{D_{ik}}   ), i =j.
\end{array} \end{equation}


%The Onsager symmetry of equation~\ref{eq:y1} is investigated in Ref.~\citenum{mwn2015}.

\textbf{Molecular dynamics, via the Onsager's hypothesis.}
In molecular dynamics, molar densities are sums of delta functions, $ c_i (\mathbf{x},t) = \displaystyle \sum_{a_i} \delta (\mathbf{x}- \mathbf{x}_{a_i}(t),$ and products or ratios of such expressions are undefined; however  Fourier transforms of those $ \hat{c}_i (\mathbf{k},t) =  \sum_{a_i} \exp ( \, \imath \mathbf{k} \mathbf{x}_{a_i}(t) ) \, , \mathbf{k} = ( \,  \frac{2 \pi n_1}{L_1},  \frac{2 \pi n_2}{L_2}, \frac{2 \pi n_3}{L_3} ) $ are smooth and their products make sense. Since in equilibrium ensemble $x$ is uniformly distributed, $\langle \exp(\,\imath\mathbf{k}\mathbf{x})\,\rangle=0$. As for autocorrelation functions,
\begin{equation}
    \mathcal{C}_{ij}(\mathbf{k},t )= \langle \hat{c}_i (\mathbf{k},t)  \hat{\bar{c}}_i (\mathbf{k},0) \rangle = \displaystyle\sum_{a_i,b_j} \langle \ e^{  \imath \mathbf{k} ( \,  \mathbf{x}_{a_i}(t)-\mathbf{x}_{b_j}(0) ) \,  }\ \  \rangle
\end{equation}

Assuming that, in statistical sense, deviations of $c$ from the equilibrium values are small, %$y_i =\frac{c_i}{c^T}$ implies

\begin{equation}
  y_i^{(1)} = \frac{c_i^{(1)} c_T^{\infty }- c_i^{\infty }c_T^{(1)}}{( \, c_T^{\infty }) \,^2}, \ \sum_{i=1}^{n}  y_i^{(1)} =0.
\end{equation}

\noindent and thus autocorrelation functions of $y$ and  $c$ is are related by
\begin{equation}\begin{array}{l}
    \mathcal{Y} (\mathbf{k},t )= ( W \mathcal{C} W^T ),
    \\
W=\left\{ W_{ij} \right\}, W_{ij}  = \frac{1}{c_T^\infty }\cdot  ( \, \delta_{ij} -y_i^{\infty } \frac{\partial  \log  c_T^\infty}{\partial y_j} ). \ \mbox{\small $i,j  =1,2,\ldots n-1$}.

\end{array} \end{equation}
 
\noindent Autocorrelations $\mathcal{Y}$ , by the Onsager hypothesis, satisfy the linearized equations,

\begin{equation}
    R \frac{\partial }{\partial t} \mathcal{Y} = k^2 Q \mathcal{Y},
\label{eqAC}  
\end{equation}

Equations~\ref{eqAC}
%are constant coefficients ordinary differential equation (ODE),  with $t\rightarrow 0$ limit determined by thermodynamic factors, the activities coefficients. Those ODE's
can be solved by diagonalising $R^{-1} Q$, with the rate of solutions decay determined by eigenvalues of $k^2 R^{-1} Q,$ $t\rightarrow 0$ limit determined by thermodynamic factors, the activities coefficients. Those eigenvalues in turn can be expressed via relative diffusion coefficients and thermodynamics factors appearing in $Q$ and $R.$ Using those relations, the relative diffusion coefficients and activities can be found \cite{mwn2015}.

%added , Jan 21, 2021 brief intro to binary electrolytes and concentrated solution theory
\noindent {\bf Electrolytes continuum model parameters}

In the Newman's concentrated solution theory\citenum{Newman2004}, an important part of the widely accepted and used Doyle-Fuller-Newman (DFN)  model\citenum{Fuller1994a},
%\cite{NewmanBook}
Stefan-Maxwell relative diffusion coefficients play a key role, and together with thermodynamic quantities (activity coefficients and partial volumes) determine other model parameters. For binary electrolytes with $+,-,0$ species, electrolyte diffusion coefficient $\mathcal{D}$, transference number $t_+^0$, and apparent conductivity $\kappa$ are given by:
\begin{equation} \begin{array}{l}
    \displaystyle \mathcal{D} = \frac{\left( z_+ - z_- \right) D_{0 +}  D_{0 -} }{z_+ D_{0 +} - z_- D_{0 -}}, \
     \displaystyle t_+^0 = \frac{ z_+ D_{0 +}  }{z_+ D_{0 +} - z_- D_{0 -}},
     \\
 \displaystyle \frac{1}{\kappa} = \frac{ R T \nu_+ \nu_- \overline{V}_0}{\left( F z_+ \nu_+\right)^2 y}\left( \frac{y}{D_{+-} } +  \frac{1 - \nu y }{\nu_- D_{0+} + \nu_+ D_{0-}} \right) \left( 1+ \left( \frac{\overline{V}_e}{\overline{V}_0 } - \nu \right) y \right),
    \label{cp diff_def}
\end{array} \end{equation}
\noindent where  $\nu_+$ and $\nu_-$ are electrolyte stoichiometric ion coefficients, $y, y_0$ are salt and solvent molar fractions,  $\frac{y_+}{\nu_+}= \frac{y_-}{\nu_-} =y$, $y_0 + \nu y =1$, $\nu = \nu_+ + \nu_-,$ $z_+, z_-$ of the electrolyte ions, $\overline{V}_e = \nu_+ \overline{V}_+  + \nu_- \overline{V}_-$ is partial electrolyte volume.

Computation of Stefan-Maxwell diffusivities based on Onsager's decay of fluctuations hypothesis for LiPF$_6$ in PC and EC/DMC solvents was performed by \citeauthor{2002PhDWheeler}, using a simple ball-and stick binary electrolyte model for solvent molecules with Lennard-Jones and Coulomb interaction potentials for solvent atoms and LiPF$_6$ ions, \cite{2002PhDWheeler}. Recently, a careful experimental measurement of all continuum modeling parameters for  LiPF$_6$ PC solvent  was performed in \cite{Hou2020}; experimental results are quite close to the computed values by Wheeler.\cite{2002PhDWheeler}.

\end{document}

